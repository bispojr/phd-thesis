
% resumo em português
% \begin{resumo}[Resumo] 

% Texto texto texto texto texto texto texto 
% % \noindent %- o resumo deve ter apenas 1 parágrafo e sem recuo de texto na primeira linha, essa tag remove o recuo. Não pode haver quebra de linha.

%  \vspace{\onelineskip}
    
%  \noindent
%  \textbf{Palavras-chaves}: Texto. Texto. Texto. Texto. Texto.Texto.
% \end{resumo}



% resumo em inglês
\begin{resumo}[Abstract]
%\begin{otherlanguage*}{english}

\glsfirst{CSE} concerns the reflection of equitable variables like gender, race/ethnicity, socioeconomic status  and culture. However, addressing how to balance different sources of inequities is still an open challenge. Although \gls{CAPE} framework can map most of the main variables to an equity analysis, the concept of capacity is strongly related to resources, ignoring some essential aspects relative to the real opportunities for a computing student. Another framework that can address this problem is the \gls{CA}, having the freedom of being educated is one of the aims of this perspective. In this direction, active learning and \gls{SDL} can potentialize the freedom of \gls{CSE} students, promoting more autonomy and crucial soft skills in our complex society. \gls{SDL} is a potential equitable practice, but there are open challenges to consider regarding when and how to use it. Understanding better how \gls{SDL} effectively occurs in \gls{CSE} students can also contribute to comprehending the potentiality of active learning in terms of capabilities. Thus, this doctoral research investigated how do \gls{CSE} students conduct their \gls{SDL} in developing countries from the \gls{CA} lens. Three \glspl{RG} help to address this question in a qualitative approach: (\gls{RG}1) understanding how \gls{CSE} students build their \gls{SDL} trajectories in developing countries; (\gls{RG}2) mapping the main elements of \gls{SDL} capabilities observed in \gls{CSE} students in developing countries; and (\gls{RG}3) recommending guidelines to (\gls{CSE}) educational stakeholders concerning how to consider effectively equity issues and active learning from the \gls{CA} lens.  The results are structured over the perceptions of two \gls{CSE} Brazilian undergraduates about their \gls{SDL} trajectories, being each one from the lowest and highest \gls{SES} of their class, respectively. Interviews and other data sources helped to better situate the findings. The doctoral contributions were in (i) the use of \gls{CA} as an equity theoretical framework in computing research, and \gls{CEd}, (ii) the proposition of a new concept called \gls{SDL} capabilities, and (iii) a pragmatic instantiation of equity discussions in \gls{CSE} (beyond other scientific contributions to Computing and Education in a general way).

   \vspace{\onelineskip} 
 
   \noindent 
   \textbf{Keywords}: Computing Education. Equity. Higher Education. 
   \\ \mbox{ } \hspace{48pt} Self-Directed Learning. Capability Approach.
 %\end{otherlanguage*}
 \end{resumo}