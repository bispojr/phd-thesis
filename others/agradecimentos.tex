% ----------------------------------------------------------
% AGRADECIMENTOS
% ----------------------------------------------------------
\begin{agradecimentos}

Este trabalho é fruto de uma produção coletiva de muitas pessoas. É bem verdade que, de hoje em diante, este trabalho será atribuído como sendo de minha autoria. Porém, considero ser impossível ignorar ou desprezar, durante todo o processo (e até antes mesmo desse ciclo de formação), o quanto ele foi afetado pela influência da vida de várias pessoas.

É também bem triste perceber a minha incapacidade de poder fazer com justiça os meus agradecimentos neste momento. Como serei tentado a citar alguns nomes, ficam aqui já os meus sinceros agradecimentos a todos que contribuíram de alguma maneira para que tudo isso ocorresse. Sem vocês, com certeza, eu não teria condições de dar todos esses passos à frente.

Começo agradecendo ao Criador de todas as coisas. Àquele que criou todas as coisas e, ori\-ginalmente, disse que eram boas. Agradeço a Ele por todos os incômodos e pelas inquietações que me levaram a fazer uma pesquisa em Computação diferente das que eu normalmente já havia conduzido. Que "corra a retidão como um rio, e a justiça, como um ribeiro perene" (Profeta Amós 5.24).

Também agradeço à minha família, meu lar. O carinho e cuidado que recebo da minha esposa Ileamá e da minha pequena Larissa. Esse projeto foi uma decisão coletiva nossa e agradeço a Deus por ter essas duas companheiras de jornada que enchem a minha vida de alegria e esperança.

Também agradeço a todos os meus familiares, sempre me incentivando a explorar todo o potencial que existe em mim. Em especial: (i) aos meus irmãos Tarik e Polly. Amo vocês. Tenho certeza de que papai e mamãe estariam orgulhosos do ser humano que nos tornamos hoje; e (ii) aos meus sogrinhos, Seu Ironcil e Tia Ianê... vocês moram de graça no meu coração.

Agradeço à Universidade Federal de Jataí (UFJ) que acreditou no meu trabalho e contribuiu efetivamente para que esse ciclo da minha formação fosse concluído. Ficam os agradecimentos especiais a todos os membros do Bacharelado em Ciências da Computação da UFJ.

Agradeço à Coordenação de Aperfeiçoamento de Pessoal de Nível Superior (CAPES) pela bolsa concedida durante a maior parte do doutorado. Tanto a bolsa no Brasil quanto no exterior foram fundamentais para que essa pesquisa fosse conduzida com a devida qualidade.

Agradeço aos meus dois orientadores: Profa. Simone Santos (Centro de Informática), Prof. Marcus Vinicius Matos (Brunel University London). Duas coisas são motivos de agradecimentos para vocês dois: confiança e paciência. Obrigado por confiarem no meu trabalho, nas minhas ideias e por permitirem essa caminhada junto a vocês. Obrigado pela paciência, principalmente nos momentos em que não havia muita clareza sobre o objeto de pesquisa, mas houve compreensão e o tempo necessário para que essa "filha" nascesse. Também devo registrar meus agradecimentos aos meus orientadores nos meses iniciais de chegada no programa de doutorado:  Prof. Sérgio Soares (Centro de Informática), e Profa. Liliane Fonseca (Universidade Católica de Pernambuco). Agradeço por me acolherem e me acompanharem nesse primeiro momento de chegada por aqui.

Também expresso a minha gratidão a todos os colaboradores do Centro de Informática. 
Ter acesso ao laboratório 24 horas por dia e contar com toda a infraestrutura, com todo o apoio operacional e administrativo, trouxeram a mim paz e condições reais para que o doutorado pudesse chegar a esse momento final.

Também não poderia deixar de agradecer a todos os membros do Grupo NEXT de pesquisa pela caminhada e camaradagem nesses quatro anos de doutorado. Deixo meus agradecimentos especiais a dois "irmãos de orientação": Davi Maia e André Ribeiro. Agradeço pelas várias discussões e boas reflexões sobre Educação em Computação e sobre a vida. Força e sucesso! Vamos que vamos.

Agradeço a todos os meus companheiros de luta de laboratório da APG. Em especial, a David Cavalcanti, Yeda Lima e Susi Vila Nova. Dividir o lab e a dureza da caminhada de uma pós-graduação com vocês tornou essa jornada menos dura. Obrigado pelo companheirismo de vocês.

Ficam também os meus agradecimentos à toda a comunidade de Educação em Computação no Brasil, em especial à CEduComp-SBC. Ser parte dessa comunidade e de sua história é um motivo de imensa alegria para mim. Deixar alguns nomes preciosos fora desses agradecimentos é difícil (e deixarei, infelizmente). Mas eu tenho que fazer um registro em especial ao meu amigo cientista Robson Feitosa. Bom poder caminhar essa jornada doutoral ao seu lado, meu querido. Um abraço pra todo esse povo bonito do IFCE Crato.

Tenho que deixar um registro especial para esse meu casal de padrinhos: Kathleen Vasconcelos e Jônatas Abreu. As nossas muitas conversas sobre família, Reino de Deus e justiça nos levaram ocasionalmente aos textos de Amartya Sen. Muito obrigado pelas conversas e pela boa compa\-nhia nessa caminhada. Tenho certeza absoluta que a construção do meu referencial teórico não existiria nesse formato sem as boas risadas e discussões importantíssimas sobre equidade.

Também agradeço aqui a três comunidades que acolheram a mim e à minha família com grande carinho na Inglaterra. Primeiro à toda comunidade internacional de estudantes evangélicos, mas especialmente aos brasileiros "sempre ABUenses" que vivem por essas bandas: (i) meu agradecimento a Marcos Vinícius, Priscila, e Aurora que receberam tão carinhosamente em sua casa em Uxbridge em um momento tão difícil de perda familiar... meu muito obrigado de coração; (ii) também agradeço ao carinho de Miguel e Giovanna que também abriram carinhosamente a casa deles para receberem a minha família em Londres; e (iii) aos demais que dividiram momentos de conversas, risos, e apreensões (especialmente Henrique e família) e aquele churrasco brasileiro que nunca é demais [risos] (Juliana Noronha e família).

Em segundo, agradeço à toda a comunidade do Exército da Salvação em Cradley Heath. Embora estrangeiros, fomos recebidos como se não fôssemos nessa comunidade. Meu agrade\-cimento especial é para Heatham por todo o seu carinho e cuidado que certamente vem do Reino dos Céus. E, por fim, também agradeço à toda a comunidade do \textit{Dudley \& Sandwell Table Tennis Club}, em especial a Chris Dunkley por me receber tão carinhosamente nos treinos no \textit{African-Caribbean Community Network} e nos jogos na liga inglesa de tênis de mesa.

\end{agradecimentos}
