\chapter{Interview Excerpts}
\label{chap:interview-excerpts}

This section presents the interview excerpts in Brazilian Portuguese (in their original transcript before translation). The excerpts used in this work (see Section \ref{chap:results}) were translated to English for a better reading flux. Section \ref{interview-exc-sec:chavo} refers to Chavo’s interview excerpts, and Section \ref{interview-exc-sec:quico} to Quico’s ones. 

\section{Chavo’s Interview Excerpts}
\label{interview-exc-sec:chavo}

The original Chavo's answers in Brazilian Portuguese concerning 
% \gls{IQ}.1, 
\gls{IQ}.2, 
\gls{IQ}.3, 
% \gls{IQ}.4, 
\gls{IQ}.5, 
\gls{IQ}.6, 
%\gls{IQ}.7, 
\gls{IQ}.8, 
\gls{IQ}.9, 
\gls{IQ}.10, 
\gls{IQ}.11,   
\gls{IQ}.12, and 
\gls{IQ}.13
% \gls{IQ}.14 
questions are presented as follows. The other answers have not been explored in an appropriate way yet.

% \subsection{Chavo’s IQ.1 answer}
% \label{interview-exc-ss:chavo-iq1}

\subsection{Chavo’s IQ.2 answer}
\label{interview-exc-ss:chavo-iq2}

\begin{quote}
    \itshape
    "Eu acho que eu tive muita expectativa com administração. Eu sabia que tinha, mas não sabia muito como era... eu nunca tinha visto muito. A parte de programação eu já tinha estudado um pouco antes, já tinha vivido um pouco antes. Então eu já sabia um pouco o que iria acontecer. Mas, tipo... me chamou muito a atenção, porque eu esperava que, tipo... eu não sei como é a administração, eu sei que vai ter questão de gerenciar alguma coisa de projeto, ciclo de vida de projeto em alguma cadeira, mas não sabia como seria. Aí que eu iria ter que estudar, tipo... os primórdios da administração, administração científica, essas coisas, sabe? Aí isso mudou muito".

    \colorbox{black!15}{Eu: Sim. Aí tu não tinha percepção clara e isso aí veio durante o curso?}

    "Isso". 
\end{quote}

\subsection{Chavo’s IQ.3 answer}
\label{interview-exc-ss:chavo-iq3}

\begin{quote}
    \itshape
    "Hoje mudou muita coisa. Eu pretendo muito me aprofundar na parte tanto de programação, sim..., mas também na parte de nuvem, que é algo que eu estou vendo mais agora e eu me interessei bastante. Sempre me interessei pela parte de segurança. Então [essas] são [algumas] das áreas que eu gosto muito... e programação. Eu sempre gostei de programação. E a minha expectativa é tentar correr atrás para tentar me desenvolver: virar [desenvolvedor] júnior, pleno, sênior, e conseguir me aprofundar nessas áreas que eu gosto para tentar ser um dos especialistas no mercado, ser referência. Acho que a gente sempre quer ser referência no que a gente gosta".

    \colorbox{black!15}{Eu: Sim, lógico. E aí, bom... a partir da resposta que você deu aí, tem principal-} \colorbox{black!15}{mente na área de SI... tem esses dilemas daquele que 'coda' e aquele que gere, e} \colorbox{black!15}{tem os que são híbridos: os que curte tanto um quanto outro. Mas, assim... você} \colorbox{black!15}{tem afinidades mais técnicas? Como é que você se enxerga assim nesse universo} \colorbox{black!15}{aí? Porque você tem competências mil aí... em jogo aí no curso de SI [Sistemas} \colorbox{black!15}{de Informação]. Como é que você enxerga esse negócio aí?}

    "Então, eu nunca fui... Foi um pouco mais difícil para mim a parte de gestão. Eu estou conseguindo melhorar bastante e agora, principalmente agora na cadeira. Me perdi um pouco, mas agora estou voltando... estou pegando a manha de como que realmente a gente faz para conseguir gerir as coisas direitinho. Acho que a parte técnica sempre foi algo que já era mais fácil para mim, porque como eu já tinha estudado um pouco de robótica antes, no ensino médio, algo que me ajudou muito, então a parte técnica foi muito boa. Aí a parte de gerenciar está vindo mais agora. Agora que eu estou conseguindo melhorar ela".      
\end{quote}

% \subsection{Chavo’s IQ.4 answer}
% \label{interview-exc-ss:chavo-iq4}

\subsection{Chavo’s IQ.5 answer}
\label{interview-exc-ss:chavo-iq5}

\begin{quote}
    \itshape
    "Sim. Geralmente, a minha semana... Foi uma loucura também o que aconteceu. Porque eu preenchi o formulário, quando eu coloquei só tinha eu e minha mãe, e aí eu consegui entrar agora em uma empresa de tecnologia. Aí agora já mudou os dados do formulário. Mas está tranquilo. 
    
    Aí, tipo... agora eu trabalho. Acordo de 06h00, 06h20 mais ou menos. Aí eu fico até... assim... descanso um pouquinho, estudo, tomo café. Aí entre as 06h00 e as 10h00. 10h00 eu começo a trabalhar. Aí para de meio-dia, volto 01h00. Aí de 01h00 até 04h00, que é o estágio que eu consegui com a Federal, aí são 6 horas.

    Aí quando termina às 04h00, eu largo, eu me arrumo literalmente. É um pouquinho corrido, mas tranquilo. Eu me arrumo... aí eu tenho duas opções: ou eu vou andando, porque é perto, aí dá para ir andando tranquilo, gasta mais ou menos uns 20, 25 minutinhos.

    \colorbox{black!15}{Eu: Pertinho.}
    
    É um tempinho. Ou eu pego o ônibus. Só que o ônibus ele passa de 04h20. Aí tem a correria de se arrumar, torcer para ele passar às  04h25 ou 04h20, mas dá para pegar tranquilo. 
    
    Aí geralmente é isso. Tenho aula da faculdade. Quando largo, pego o ônibus no CCN [Centro de Ciências Exatas e Naturais] e vou para casa. Aí eu só reviso as coisas e vou dormir... comer e dormir.
    
    A semana é mais ou menos assim. Final de semana eu só olho mais para ver as coisas da faculdade e organizar um pouco as coisas da semana... ver se tem algum compromisso, se precisa ir para a empresa ou se precisa ir para a faculdade para estudar mais... essas coisas. E aproveitar. De resto, é mais isso.

    
    \colorbox{black!15}{Eu: E aí perguntar a tu, o estágio que tu está agora, tu entrou agora recente? \ } \colorbox{black!15}{Faz quanto tempo que tu entrou?}
    
    Eu acho que faz mais ou menos umas duas semaninhas, acho que no máximo. Essa eu acho que vai fazer duas semanas, eu acho. Faz pouquíssimo tempo que eu entrei.

    \colorbox{black!15}{Eu: Massa. Beleza. Depois a gente volta para essa parte do... Beleza? Legal?}
    
    Tranquilo.
    
    \colorbox{black!15}{Eu: E aí deixa eu te perguntar. Nesse trânsito aí... nesse processo aí... de tu sair} \colorbox{black!15}{de um lugar para o outro. Como é que gere a alimentação, comida? Como tu faz} \colorbox{black!15}{esse processo aí?}
    
    Geralmente, eu e a minha mãe, a gente geralmente divide. Às vezes, ela faz o almoço. Quando ela está sem tempo, eu desenrolo. Aí a gente deixa na geladeira. Eu sempre deixo separado a marmita às vezes. Quando eu preciso ir para a faculdade, ir para o trabalho... já tem a comida separada. Aí quando eu sei que vou ficar em casa, só deixo a comida congelada ali. Chegou a hora do almoço, esquentar... está tranquilo. 
    
    Agora, o lanche... ou eu pego algumas coisas e levo... em casa, tipo macarrão, carne, alguma coisa assim. Ou eu compro um salgado perto do \gls{CIn}. Geralmente eu faço muito isso... é mais prático".
\end{quote}

\subsection{Chavo’s IQ.6 answer}
\label{interview-exc-ss:chavo-iq6}

\begin{quote}
    \itshape
    “Quando é algo que eu não sei realmente do que eu estou vendo, primeiro eu pesquiso na internet, tento buscar mais a fundo se tem documentação. Eu gosto muito de olhar documentação ou pesquisar vídeo no Youtube. E também tem bastante coisa, tem muito material bom na internet. Aí eu, primeiro, foco nessas duas coisas. Aí tento achar e entender como é que aquilo funciona.

    Eu faço meio que assim: primeiro, eu vou tentar ler, vou tentar ver o que aquilo é, como funciona. Por exemplo, digamos que... por exemplo, passe alguma coisa relacionado a Banco de Dados. Aí agora a cadeira de Banco de Dados, aí ele tem lá… está dando um tópico de Banco de Dados Conceitual. Eu não sei o que é conceitual: eu pesquiso na internet, eu procuro nos sites que eu mais conheço relacionados a tecnologia. Como tem muitos sites que aparecem de Linux, tem o do Tech, tem no Youtube também. Aí eu pesquiso, tento estudar, aprender. Aí depois que eu aprendo, eu geralmente eu vejo os pontos, eu coloco meio é… um bloquinho com os pontos, assim, que eu aprendi, que eu coloco no computador. Ou se o professor já disponibilizar um exemplo, o assunto, aí eu assisto pelo do professor e depois eu pesquiso na internet para tentar dar uma revisada também. Eu faço meio que uma mistura dos dois.

    \colorbox{black!15}{Eu: Massa, beleza. [...]. Então, você está em uma disciplina... está dentro da fa-} 
    \colorbox{black!15}{culdade. E aí o professor te exige algo. E aí existe alguma coisa diferente que tu} 
    \colorbox{black!15}{faz porque é um assunto de faculdade ou a estratégia segue mais ou menos a } \mbox{    } \colorbox{black!15}{mesma linha?}

    Depende. Depende também do escopo que ele passa para a gente. Porque, por exemplo, na cadeira de Contabilidade, o professor passou para gente uma coisa que ele nunca tinha passado para o pessoal, que era relacionado a fazer meio que uma "aplicação" utilizando coisas gerenciais. Então, tipo... para mim, foi algo que ele não tinha passado material e que, nessa cadeira, eu tive que pesquisar por conta própria. Aí eu tive que utilizar um método diferente. Então, tipo... a partir do escopo que ele me deu, eu fui pesquisando os pontos. Aí, como é que eu poderia... por exemplo, utilizando o Python, como é que eu poderia criar uma aplicação? Aí eu pesquisei, sabe? Como ferramentas ou bibliotecas para auxiliar na criação de telas... realmente uma aplicação em Python. Aí com isso eu conseguir levantar o tópico de... tipo... qual seria a melhor para utilizar em questão do tempo do projeto, se o projeto tivesse muito apertado,  muito grande... [ou] se eu poderia utilizar uma simples ou uma mais completa. Aí, geralmente, eu utilizo esse ponto e, depois disso, depois de estudar qual o melhor ponto, eu entro mais a fundo com o tema que o professor passou. Tipo... primeiro, eu vejo geralmente os requisitos. No caso... nesse caso, acho que seriam os requisitos e depois eu vejo direitinho a parte mais profundamente para planejar quando que eu vou começar a fazer ou já começar a fazer direto". 
\end{quote}

% \subsection{Chavo’s IQ.7 answer}
% \label{interview-exc-ss:chavo-iq7}

\subsection{Chavo’s IQ.8 answer}
\label{interview-exc-ss:chavo-iq8}

\begin{quote}
    \itshape
    "Eu tento entrar em contato com os meus amigos. Assim... os que eu mais convivo geralmente. Aí, geralmente, eu pergunto para eles: `Gente, vocês entenderam o que o professor pediu?' ou `Vocês entenderam tal assunto?'. Que a gente tem o grupinho da gente, aí eu pergunto para eles.

    Acho que são mais ou menos, assim... umas 7 pessoas, mais ou menos 8 pessoas. E aí eu tento falar com o pessoal. Se eu não consigo, ou se eu não entender, eu pergunto no outro grupo que tem do pessoal da sala, que é o do período que a gente entrou, que é um pessoal que a gente está mais familiarizado desde o primeiro período. Aí eu tento perguntar para eles. Aí peço ajuda ao pessoal e o pessoa ajuda e a gente está se entendendo. E é bom, porque se tiver mais alguma pessoa com dúvida, aí a gente até ajuda... vai um Discord, vai uma ligaçãozinha. Aí a gente desenrola para conseguir entender o assunto. Geralmente, eu faço isso". 
\end{quote}

\subsection{Chavo’s IQ.9 answer}
\label{interview-exc-ss:chavo-iq9}

\begin{quote}
    \itshape
    "Sim. Eu sou muito tranquilo de trabalhar em grupo, consigo tranquilamente... eu consigo me adaptar. Qualquer grupo que você conseguir me colocar, assim... eu consigo me adaptar com o pessoal. Isso é uma coisa que para mim é de boa. Mas, assim... eu gosto muito de fazer trabalho com o pessoal que eu já tenho mais afinidade, porque fica mais tranquilo, porque o pessoal já conhece mais sobre mim, sobre a minha rotina também e eu conheço também do pessoal. Aí fica muito mais tranquilo, porque a gente já sabe, a gente conhece mais um ao outro, então sabe... tipo... `Ah, vamos fazer tal parte, outro faz tal parte'. A gente decide direitinho e fica melhor. Mas se não der, eu... Qualquer grupo, assim... com o pessoal, que me colocar, eu consigo desenrolar. Principalmente com o pessoal do período que a gente está, desde o primeiro. Eu tenho afinidade muito boa com todo mundo, então qualquer grupo é tranquilo".
\end{quote}

\subsection{Chavo’s IQ.10 answer}
\label{interview-exc-ss:chavo-iq10}

\begin{quote}
    \itshape
    "Eu já usei bastante a biblioteca física quando eu estava procurando livro de algoritmo. Porque livro de algoritmo? Como tinha muita coisa na internet, mas algumas coisas específicas você não encontrava na internet ou, tipo... só encontrava se você pesquisasse, por exemplo, em inglês. E fosse procurando bem muito e você conseguia achar. E tem outros que tinham em livro, aí eu achei mais fácil. Tipo... tem principalmente livros de referência, de algoritmos e estruturas de dados, acho que tem um que tem mil páginas, só que eu não lembro agora o autor, mas sei que tem tudo ali. Aí eu pegava ele. Via se tinha na biblioteca do CCEN [Centro de Ciências Exatas e da Natureza da \gls{UFPE}]. Se não tivesse, eu baixava ou... porque o livro é meio carinho. Aí eu dava uma desenrolada. 
    
    Mas eu gosto de usar o livro quando é um pouquinho mais difícil de encontrar na internet ou quando tem alguma coisa muito específica que o professor fala: 'Vocês não vão conseguir encontrar na internet' ou vai ser mais difícil de encontrar. Aí eu gosto de pegar um livro de referência aqui... já facilita mais as coisas.".    
\end{quote}
\subsection{Chavo’s IQ.11 answer}
\label{interview-exc-ss:chavo-iq11}

\begin{quote}
    \itshape
    "Bom, acho que tem sim dois lugares que eu mais gosto de estudar, no meu quarto principalmente. É porque geralmente eu fico mais por aqui, então como eu fico em casa a maior parte sozinho, eu já estou acostumado e não tem problema. Mas quando eu tenho que ir para a faculdade ou, por exemplo... ou no trabalho, no escritório... dá para conseguir trabalhar de boa, dar uma revisada de boa, mas na faculdade principalmente. 
    
    Eu estou fazendo muito isso essa semana. Vou fazer amanhã que amanhã vai ser corrido demais. E perto da gente, do [local de aulas de] SGE ali, que é do lado literalmente, tem uns banquinhos ali, tem tomada. Tipo... eu gosto de lá porque lá é ventilado e, mesmo com algumas pessoas, o pessoal respeita o silêncio.     É tranquilo de estudar por ali, ou então na biblioteca. Só que a biblioteca, como fica um pouco mais distante, eu prefiro já ficar pelo \gls{CIn} mesmo por causa do Wi-Fi.
    

    \colorbox{black!15}{Eu: Então você consegue fazer em casa algumas coisas. E tem algo que tu faz em}
    \colorbox{black!15}{casa e que não consegue fazer na faculdade, ou tem coisas que você na faculdade}
    \colorbox{black!15}{e que não consegue fazer em casa, em termos de estudo... Assim?}
    
    Acho que um pouco... é um pouco. É porque é um meio termo, não é? Porque, assim... em casa, às vezes, você procrastina um pouco. Isso acontece comigo relativamente sempre. Aí, tipo... na faculdade, eu consigo ter um foco maior, de verdade. Eu consigo me concentrar e ficar mais focado por mais tempo. Em casa, eu tenho algumas distrações, mas estou trabalhando para tentar melhorar. Mas acho que, em si, é mais isso. O ruim só é que, como eu ainda estou... o meu fone de ouvido BlueTooth está com a bateria ruim, eu estou esperando outro chegar, e está ruim de ver vídeo. Mas fora isso, na faculdade, quando eu estou sozinho, por exemplo, em um cantinho estudando... eu tenho um pouco mais de foco por mais tempo do que ficar em casa em si, porque como tem, por exemplo, WhatsApp e tem mensagem do pessoal, muitas coisas. Então eu estou em casa, é só levantar, ir ali e voltar. Mas, fora isso... acho que é mais isso mesmo".
\end{quote}
\subsection{Chavo’s IQ.12 answer}
\label{interview-exc-ss:chavo-iq12}

\begin{quote}
    \itshape
    "Entendo. Acho que não... pelo estudo... Eu consigo... dá para conseguir ficar com estudo e sem trabalhar, por causa que é Federal, mas é um pouquinho mais complicado. Porque como é só eu e mainha em casa... e antes de eu trabalhar ficava mais difícil por causa da passagem do ônibus e da comida na faculdade. E isso daí, porque... querendo ou não, passagem gasta bastante e comida também. Mas assim... eu acredito que seria mais isso. Mas é porque antes... Espera, eu acho que eu me... Espera aí. Tu pode repetir a pergunta, por favor? Porque eu me perdi. 

    \colorbox{black!15}{Eu: Posso. Eu estou falando o seguinte, se é essencial o seu trabalho pensando na} 
    \colorbox{black!15}{garantia do seu estudo.}

    Entendi. Certo. Pronto... Seria mais a questão do transporte e um pouco da alimentação. Porque, assim, dá para conseguir ficar... dá. Só que você tem que realmente ser bem apertado. Por exemplo, antes de eu começar a fazer o estágio, o que é que eu fazia? Eu fazia muito assim... como a ida para a faculdade é tranquilo, ainda é de dia e tal... tudo certinho. Eu vou muitas vezes andando, porque é próximo, aí já economiza uma passagem. Eu só iria [de ônibus] na volta. Um exemplo, assim. Então é algo que você, assim... dá para conseguir desenrolar sem, mas fica bem mais complicado, sabe?".
\end{quote}
\subsection{Chavo’s IQ.13 answer}
\label{interview-exc-ss:chavo-iq13}

\begin{quote}
    \itshape
    “Entendi. Isso já aconteceu comigo já no início, quando eu entrei no curso. No início, foi na cadeira de P1 [Programação Introdutória]. E pesou mais em algoritmos. Mas foi algoritmo, aí, que eu consegui construir uma fonte boa de lógica de programação. Que tinham algumas listas que… tipo... mesmo eu pesquisando, mesmo eu lendo a questão, eu não conseguia entender, porque eu ainda não tinha... eu não estava conseguindo entender realmente, como tu falou. Pelo ponto de falta de... que eu não conseguia. Aí quando eu não conseguia, eu pedia muito ajuda, eu pedia ajuda ao pessoal, principalmente do pessoal que eu já conhecia ou então o pessoal da sala que eu tinha conhecido na época. Aí eu pedia ajuda. O pessoal, sempre, tipo assim: "Ah, eu consigo te ajudar", tal pessoa, tipo… aí o pessoal explicar ou então eu pedia para entrar no Discord para ajudar, e eu pesquisava bastante. Então, tipo, eu ia dormir um pouquinho mais tarde, mas eu tentava pesquisar, tentava estudar de novo para entender melhor aquilo ali que eu não tinha conseguido”. 
\end{quote}
% \subsection{Chavo’s IQ.14 answer}
% \label{interview-exc-ss:chavo-iq14}

\section{Quico’s Interview Excerpts}
\label{interview-exc-sec:quico}

The original Quico's answers in Brazilian Portuguese concerning 
 \gls{IQ}.1, 
% \gls{IQ}.2, 
 \gls{IQ}.3, 
% \gls{IQ}.4, 
% \gls{IQ}.5, 
\gls{IQ}.6, 
%\gls{IQ}.7, 
\gls{IQ}.8, 
\gls{IQ}.9, 
\gls{IQ}.10, and 
\gls{IQ}.11  
% \gls{IQ}.12, 
% \gls{IQ}.13, 
% \gls{IQ}.14 
questions are presented as follows. The other answers have not been explored in an appropriate way yet.


 \subsection{Quico’s IQ.1 answer}
 \label{interview-exc-ss:quico-iq1}

 \begin{quote}
     \itshape
     "De começo, assim... eu tentei duas vezes o ENEM. A primeira [vez], infelizmente, não deu. A nota não foi suficiente. Aí a segunda, eu ainda estava com a mesma mentalidade de fazer Ciência da Computação. Só que aí vi que, pela nota, ali não ia rolar, e Sistemas da Informação a nota dava para entrar. 

    Aí a gente passou um tempo lá, eu e painho, conversando e tal. A gente via o histórico do SISU, as notas... como é que tinha ficado, se ia dar para entrar ou não. Aí, pela análise que a gente fez, Sistemas da Informação era um curso que estava dentro do que eu queria, que era relacionado a programação, e também dava para eu entrar.

    Então, assim... esse foi o principal motivo de eu ter escolhido Sistemas da Informação. Mais assim... por abordar o que eu queria e também estar dentro ali do que a nota permitia, entendeu? Então foi mais assim, foi esse motivo a escolha.

    \colorbox{black!15}{Eu: E essa parte de programação, tu já conhecia alguma coisa ou não? Ou tu era} \colorbox{black!15}{meio 'zerado'? Como é que era essa ideia?}

    Eu tinha uma base, assim... não tão grande, mas eu já tinha visto alguma coisa, sabe? Eu já tinha visto por fora, assim... no Youtube já tinha visto coisa relacionada a programação já. Eu foquei em uma lá, que era Python. E depois eu fiz um curso técnico no IF [Instituto Federal] de Paulista lá. Foi um ano e meio, fiz Manutenção e Suporte em Informática. Aí tinha também... tinha programação. Aí já fiquei com uma base maior. Teve esse período que eu fiquei aprendendo por fora, teve esse período do IF e agora o de Sistemas, não é? Então eu já tinha aí já um contexto, mesmo que simples, mas que já tinha já. 

    [...]

    \colorbox{black!15}{Eu: Que massa. Então, assim... tu fez o IF lá de Paulista e pelo menos essa ideia} \colorbox{black!15}{de diferenciar informática de Ciência da Computação tu tinha mais ou menos em} \colorbox{black!15}{mente? Ou não? Ou lá também não era tão claro assim?}

    É porque lá no do IF foi mais... foi um geralzão, entendeu? Pelo curso que eu fiz. Tinha Programação, aí tinha Redes, tinha Manutenção de Computadores, tinha Arquitetura de Computador, então era bem abrangente aquele curso. Aí a ideia de diferenciar de ciência assim para...

    \colorbox{black!15}{Eu: Diferença de maneira clara não tinha?}

    Não tinha não".
 \end{quote}

% \subsection{Quico’s IQ.2 answer}
% \label{interview-exc-ss:quico-q2}

\subsection{Quico’s IQ.3 answer}
\label{interview-exc-ss:quico-iq3}

\begin{quote}
    \itshape
    “Rapaz... eu pretendo, assim... da imagem que eu tenho, estar trabalhando em uma área que eu goste, não é? Então, assim... é a primeira coisa que vem na minha cabeça, trabalhando em uma coisa que eu goste. Assim, mais profundamente, talvez... talvez, com o passar do tempo, se eu... no período, sei lá... sei lá, no decorrer do curso... eu entrar em algum estágio e for vendo que tem áreas dentro das empresas, que é relacionado à programação e à gestão, que eu me identifico mais. Aí eu posso seguir por algumas delas e seguisse o meu futuro. Mas chutar, assim, tão claro... eu não sei, assim: "Vou estar desenvolvendo para tal empresa, fazendo tal coisa", entendeu? Não é tão claro assim... ou vou estar sendo gestor de um projeto de tal empresa, entendeu?
    
    Mas, a princípio, é isso. Estar trabalhando em alguma coisa que eu goste, em alguma empresa. No final, eu já vou estar com uma base já e ser um profissional assim.... bom, com uma certa excelência por causa da formação.  Então é mais isso. Não está tão claro ainda na minha cabeça, até porque eu não tenho nenhuma experiência ainda na área, não estou procurando estágio. Então, talvez quando eu procurar o estágio e entrasse fica mais claro, não é?"
\end{quote}

% \subsection{Quico’s IQ.4 answer}
% \label{interview-exc-ss:quico-iq4}

\subsection{Quico’s IQ.5 answer}
\label{interview-exc-ss:quico-iq5}

\begin{quote}
    \itshape
    "É, assim, eu acho que é bem... não regrado, mas é bem sempre a mesma coisa. Como eu te falei, de manhã eu tenho, de 10h00 até 14h00 eu estou livre... Não livre! Eu estou em casa, mas fazendo as coisas da faculdade, talvez alguma coisa de casa.

    Aí se resume bem a isso o dia, quando é normal, quando tem aula. Das 10h00 às 14h00, eu estou fazendo alguma coisa da faculdade. Aí quando chega perto das 15h00 eu estou me arrumando para vir. 

    Aí eu chego aqui acho que quase 14h00... 16h00, na verdade... e fico conversando com o pessoal ali embaixo. Dá a hora da aula, a gente sobe. Aí é das 17h00 até às 20h30 só com as aulas. 
    
    Então, meio que se resume a isso. Um dia atípico... eu acho que seria um dia sem aula talvez. Porque para eu vir para a faculdade é mais tendo aula, entendeu?

    Quando não tem aula, não tem muito por que eu vir, não é? Mas, assim... se não tivesse aula e eu viesse para cá, seria mais, assim... para resolver alguma coisa de projeto que demande estar todo mundo junto do grupo, entendeu? Seria mais relacionado a isso. Mas, fora isso, é mais... eu venho para a faculdade para ter aula normalmente. 

    \colorbox{black!15}{Eu: E esse fim de dia, como é que é? Tu volta... tu chega em casa mais ou me-}
    \colorbox{black!15}{nos que horas? Como é que é a vibe?}
    
    Chego mais ou menos umas 21h00. Quando eu não fico conversando aqui até 21h00, eu chego nesse horário: 21h00. Aí quando eu estou com cabeça, eu chego em casa, sei lá... dou uma olhada na aula... eu olho alguma coisa da aula. Mas, quando não, eu entro em chamada com o pessoal, vou jogar, ou então eu fico assistindo alguma coisa. Então, o final do dia se resume basicamente a isso: fazer alguma coisa da faculdade ou então entretenimento. Por exemplo, entendeu?

    \colorbox{black!15}{Eu: E tu sai de 08:30 daqui e tu chega mais ou menos umas 21h00 em casa... tu}
    \colorbox{black!15}{falou, não é? Umas 21h00?}
    
    É. 

    \colorbox{black!15}{Eu: Ou tu sai um pouco mais antes e chega 21h00 e tal? Como é que tu chega em}
    \colorbox{black!15}{casa? Tu vai de ônibus? Como é que é o teu rolê?}

    Vou de carro. Eu moro lá em $\langle$Tangamadápio$\rangle$. Quando a BR não está com engarrafamento, é uns 22 minutos, por aí.

    \colorbox{black!15}{Eu: Mas em Paulista, tu mora em Paulista Centro ou nos bairros ali de Paulista?}
    
    Eu moro no Centro, perto da UPA, tem uns prédios perto da UPA, na estrada ali.[...] Eu já vim para cá de ônibus já. Já vim para cá de ônibus, eu demorava acho que umas 1h40, por aí. Aí tem que sair mais cedo, não é? 

    \colorbox{black!15}{Eu: E mesmo com essas integrações, ainda demora esse tempo todo?}

    Rapaz, demorava. Assim... é porque de lá de casa, para eu chegar na Macaxeira, acho que era o maior percurso que tinha, porque era mais distante e tal. Aí esse é o que comia mais tempo da minha viagem, era de lá para Macaxeira. [...] Aí quando eu vinha de ônibus tinha esse tempo aí que eu ficava no ônibus. Mas aí... agora que eu tirei a habilitação, estou vindo de carro mesmo. 

    \colorbox{black!15}{Eu: É mais rock! Isso aí não tem nem o que falar, não é?}

    É bem melhor... até por questão de segurança e tal... de noite, porque você voltar e tal. Aí e bem melhor, entendeu?
    
    \colorbox{black!15}{Eu: Mas que massa. Beleza. Cara, e no final de semana tem alguma coisa típica}
    \colorbox{black!15}{que acontece assim?}

    Normalmente, às vezes, eu fico em casa mesmo. Aí eu, sei lá... como o meu horário, assim... para fazer exercício e tal. É meio difícil na semana. Porque eu estou fazendo uma coisa de manhã e de tarde eu venho para cá. Aí, normalmente, às vezes eu corro, entendeu? Ou quando eu não corro, eu fico em casa, fico jogando com o pessoal, ou então eu saio.

    Entendeu? Pronto. Esse final de semana agora eu saí, esse agora eu também vou sair no domingo. Então, às vezes tem essas saídas que eu faço no fim de semana, mas normalmente eu fico mais em casa. Normalmente eu fico mais em casa. Aí é jogando ali e tal".
\end{quote}

\subsection{Quico’s IQ.6 answer}
\label{interview-exc-ss:quico-iq6}

\begin{quote}
    \itshape
    “Assim, dependendo do projeto, do que eu vou ter que aprender, eu vejo que normalmente... assim, ... é a trilha que você tem que seguir. [...] Então, tipo, ir de onde você começa, intermediário e mais avançado. No momento, eu procuro fazer assim para ter um desempenho ali, um desenvolvimento melhor, entendeu? Um fluxo melhor de desenvolvimento. Então essa é a abordagem que eu procuro fazer. Assim, eu consigo pegar isso mais para programação. Não sei se para as matérias em si, mas programação é assim que eu faço. Quando começo a aprender alguma coisa, eu vou vendo ali o básico, que, normalmente, toda linguagem de programação tem sempre o básico que você tem que aprender. Aí eu vou conseguindo ir para outras coisas, coisas mais difíceis e assim eu vou escalando, entendeu?”.
\end{quote}

% \subsection{Quico’s IQ.7 answer}
% \label{interview-exc-ss:quico-iq7}

\subsection{Quico’s IQ.8 answer}
\label{interview-exc-ss:quico-iq8}

\begin{quote}
    \itshape
    "Mas... eu tento fazer por mim mesmo. Normalmente, eu tento fazer por mim mesmo para... Porque, assim... é uma coisa que eu estou vendo agora em... quando eu estou trabalhando... que a gente está fazendo esses projetos e a gente tem as equipes que a gente está participando e tal. Então, uma coisa que eu vejo muito. Às vezes, as pessoas fazem uma coisa, uma parte do projeto, e eu não fiz. E aí como eu não fiz, aí eu não sei aquela parte que ele fez, entendeu? Então eu tendo a fazer por mim mesmo até onde der, para eu ter esse conhecimento, entendeu?

    Porque, assim... se você não faz, você normalmente não tem. Coisa que tem que ser prática, você tem que fazer, então não tem muito o que correr. Então eu tento fazer por mim mesmo e não procurar outras pessoas para fazer aquilo, entendeu? 
    
    Mas aí entra aquilo, se eu não conseguir, aí eu falo com a pessoa, converso e tal. Como é questão de grupo e tal, eu tendo fazer aquilo para eu entender melhor, mas eu tento também envolver as pessoas para fazer aquilo junto, entendeu?".
\end{quote}

\subsection{Quico’s IQ.9 answer}
\label{interview-exc-ss:quico-iq9}

\begin{quote}
    \itshape
    "De [experiência] negativa é mais, assim... quando tem uma pessoa que não está, assim... fazendo muita coisa ali e tal. Assim... normalmente acontece quando... como está tendo o exemplo agora da disciplina de administração. A gente está em grupos que nem todo mundo a gente conhece. Nem todo mundo do grupo a gente conhece, no caso. Aí nesse cenário de pessoas que você não conhece e tal, a pessoa que não faz alguma coisa, normalmente, tende a ser um ponto negativo para mim no grupo. Entendeu? 
    
    Porque, assim... é normal. Mas aí quando tem um grupo que todo mundo se conhece, para mim, assim... a pessoa não está fazendo muita coisa ali. Como todo mundo se conhece, acaba não sendo uma preocupação, entendeu? Estar ali entre a gente mesmo, entre os amigos... então não acaba sendo uma preocupação. 

    
    \colorbox{black!15}{Eu: Há empatia também.}
    
    É, empatia... a gente às vezes entende que está passando por alguma coisa... não sabe... [alguém] está com muita dificuldade. Então, a gente acaba relevando... eu acabo relevando isso. Então, de ponto negativo é mais essa questão de um cenário que eu não conheço todo mundo. Entendeu? 
    
    E positivo é justamente isso, de você estar ali entre amigos e tal e o trabalho fluir e... não sei. É isso! De o trabalho fluir. De estar entre amigos ali... é o positivo que eu vejo, assim... em grupo. Entendeu? É mais para esse lado".
\end{quote}

\subsection{Quico’s IQ.10 answer}
\label{interview-exc-ss:quico-iq10}

\begin{quote}
    \itshape
    "Hoje, que eu consigo lembrar, são mais esses que eu falei, é mais Youtube e o ChatGPT de vez em quando, mas eu não consigo pensar em outra coisa não que eu utilize para aprender e tal. Acho que é mais isso mesmo. É muito voltado assim a... Quando eu não sei no Youtube, eu vou para o Google porque as vezes é melhor eu lendo para entender aquilo do que eu escutando alguém falar, entendeu?

    E quando eu estou lendo e não entendo, aí já é o inverso, eu vou para o Youtube para entender alguém falar, é o melhor. Aí fica meio que nisso. É bem esse o escopo que eu estou usando na metodologia para aprender e tal... é bem mais isso. Fica bem nesse mundo assim e tal.
    
    \colorbox{black!15}{\textbf{Eu}: Que massa. Beleza. Deixa eu te perguntar agora uma coisa sobre essa febre}
    \colorbox{black!15}{aí do ChatGPT, que está bombando e tudo. Eu não tenho problema nenhum com} \colorbox{black!15}{ele. Se você souber usar bem, está de boa... tranquilaço. Mas principalmente pelo} \colorbox{black!15}{tipo de coisa que eu estou de olho, aprendizagem por conta própria... tem pergun-} \colorbox{black!15}{tas que você faz no Google que o ChatGPT te dá outros nortes. E o poder dele é} \colorbox{black!15}{maior para poder te apontar insights e outras coisas. A pergunta que eu lhe faço} \colorbox{black!15}{é a seguinte: Quais são as perguntas mais típicas que tu costuma fazer para ele?} \colorbox{black!15}{Que isso para mim importa.}
    

    Estou entendendo. Assim, eu acho que o que eu pergunto mais é coisa que eu tenho que responder de atividades e tal. Então, pronto. Vamos supor, essa atividade aqui eu tive que fazer, de GPN [Gestão de Processos de Negócio]. Eu tive que fazer um relatório e tinha tópicos lá para eu desenvolver o relatório. Aí o que é que eu fiz? Como ela queria que a gente pegasse em artigos, eu fui pesquisando artigos e não estava vindo a coisa que eu queria. Aí o que é que eu fiz? Eu botei lá e pedi para ele desenvolver um parágrafo relacionado a um tópico que tinha lá. Beleza. Aí fiz isso, li lá, aí eu `Pô, beleza'. Capturei o que eu queria dali e fui ver... sem artigo. Fui ver no Google... normal. Aí lá eu vi que batia coisas e tal, entendeu? 
    
    Então, às vezes, o que é que eu faço? Eu tenho uma pergunta lá. Às vezes eu não conseguia achar o que eu queria no Google, [então] eu jogo lá no ChatGPT, pergunto para ele, ele me dá um norte. Aí, beleza. Li lá, eu vou ver de novo no Google para ver se tem a relação, entendeu? Porque pode ser que o que esteja falando lá não é verdade, então não está relacionado. Aí é bem isso. Eu pergunto lá o que eu preciso, vejo lá e vou dar uma lida para dar uma complementada, entendeu?".
\end{quote}

\subsection{Quico’s IQ.11 answer}
\label{interview-exc-ss:quico-iq11}

\begin{quote}
    \itshape
    “Então... como eu passo um bom tempo em casa, e no período da manhã e da tarde... é mais assim no meu quarto, reservado ali, e no computador, estudando. Assim, é uma coisa que eu não sei que tem a ver com a questão de você aprender sozinho, mas, quando você está ali, às vezes fica muito maçante, se você faz aquilo todo dia, entendeu? A mesma coisa... fica muito maçante. Então, às vezes, eu procuro... assim... eu acordo de manhã e não vou para o computador pra ver alguma coisa. Eu fico em outro lugar da casa, fazendo qualquer outra coisa, para... sei lá... espairecer a mente, para não ficar sempre naquela mesma coisa. Então, mas, assim, de lugar físico que eu fico em casa, é no quarto. 

    E relacionado aqui à faculdade, é o GRAD, os laboratórios que têm computadores. Então, quando eu vou precisar estudar alguma coisa, assim... não agora, mas no começo... no primeiro e segundo período ia muito para lá, quando eu chegava. Eu chegava, ia para o GRAD, ia fazer alguma coisa que precisava fazer de programação e tal, das cadeiras. Então, aí era bem no início. Ou, quando eu estou em casa, lugar físico é mais o meu quarto, e quando eu estou aqui é o GRAD.”
\end{quote}

% \subsection{Quico’s IQ.12 answer}
% \label{interview-exc-ss:quico-iq12}

% \subsection{Quico’s IQ.13 answer}
% \label{interview-exc-ss:quico-iq13}

% \subsection{Quico’s IQ.14 answer}
% \label{interview-exc-ss:quico-iq14}
