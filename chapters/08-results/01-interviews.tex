\section{Interviews from SDL Perspective}
\label{res-sec:interviews}

The coding process of interviews was conducted using Notion tool\footnote{Available in \url{www.notion.so}.}. The Quico and Chavo interview transcripts were divided into ``chunks'' with the aim of better structuring and visualizing the codes. I adopted descriptive coding \cite[p.~4]{saldana:2013} using a mixed approach (inductive and deductive simultaneously), categorizing each code group from the six steps of Knowles' \gls{SDL} model (see Section \ref{sdl-models-ss:linear}). This process generates the following six sections (Sections \ref{results-ss:strategy} to \ref{results-ss:evaluation}). All categories and codes originated from this coding process is presented in Frame \ref{frame:categories-codes}.

\begin{quadro}
    \caption{Categories and codes from coding process of the Chavo and Quico's interviews.}
    \label{frame:categories-codes}
    \centering
    \begin{tabular}{|l|l|}
    \cline{1-2}
    
    \textbf{Categories} & 
    \textbf{Codes} \\
    \hline

    \multirow{3}{*}{Strategy} &
    Iterative Process | Linear Process | Internet Technologies \\
    & Reading | Refining | Recapping | Asking People \\
    & Complexity Levels | Study Place \\
    \hline

    \multirow{2}{*}{People as Resource} &
    Friends | Classmates | Closer People \\
    & Adaptability | Empathy | Doing Alone Preference\\
    \hline
    
    \multirow{2}{*}{Non-Human Resource} &
    Digital Resources | Physical Resources \\
    & Library | Large Language Models \\
    \hline

    \multirow{2}{*}{Place as Resource} &
    Home | University | Work \\
    & Procrastination | Computing Laboratory \\
    \hline

    \multirow{2}{*}{Time as Resource} &
    Job | Working Hours | Livelihood \\
    & Transportation | Household Activities \\
    \hline

    \multirow{2}{*}{Evaluation} &
    Checking with People | Labor Market Absorption \\
    & Levels of Progression | To Do List \\
    \hline

    \end{tabular}
      \par\medskip\ABNTEXfontereduzida\selectfont\textbf{Source:} Created by the author (2024). \par\medskip
    \end{quadro}

\subsection{Strategy}
\label{results-ss:strategy}

When asked about their \gls{SDL} strategy (\gls{IQ}, n. 6, Appendix \ref{chap:appendix-scripts}), Chavo and Quico presented their perceptions. Chavo answered as follows: 
\begin{quote}
    “When there is something that I don't really know what I'm seeing, I first search for it on the Internet, and I try to look deeper to see if there is [any] documentation. I like a lot to see documentation or search for videos on Youtube. And there also are many things. There are many good materials on the Internet. So, I first focus on these two things. So I try to find out and understand how it works. 
    
    I do [it] more or less in this way: first, I'm gonna try to read, I'm gonna try to see what is, how that works. For example, let's say that... for example, some subject is required... that is related to Databases. So now the Database course is approaching the topic of Conceptual Databases. I don't know what ``conceptual'' is: I search on the Internet, and I look for sites that I know more related to technology. As there are many sites that appear about Linux (there is that called ``Tech''), there is one on YouTube too. So I research, I try to study, learn. So after I learn, I generally see the points, I put in... a notepad with the topics. As well as I learned, I put in the computer. Or if the professor has already provided an example... the subject. So I watch from the professor [video] and after I research on the internet to try to review too. I do [this] like a mix from the two [ones]\footnote{See the original excerpt in Brazilian Portuguese in Appendix Section \ref{interview-exc-ss:chavo-iq6}.}”.
\end{quote}
Chavo created an iterative process for his \gls{SDL} consisting basically of three stages: (i) reading materials, (ii) refining through watching videos, and (iii) recapping from topics. These stages were strongly assisted by Internet technologies like YouTube and reliable sites (e.g., technical content pages and professor’s blogs). He mentioned that his strategy is sensible to context, leading to different approaches when the learning needs are different. 

One of the signals that his strategy is not working is the high number of asking friends for help. When asked about his limitations during the \gls{SDL} journey (\gls{IQ}.13, Appendix \ref{chap:appendix-scripts}), Chavo answered as follows:
\begin{quote}
    “I got it. This had already happened to me in the beginning when I entered the program. In the beginning, it happened in the P1 course [Introductory Programming]. And it weighed more in Algorithms. But it was in Algorithms that I've got build a good source of programming logic. There were some lists... like... even I was researching, even I was reading the question, I couldn't understand because I still hadn't... I wasn't getting to understand in fact, as you said. Due to the lack of... that I didn't get. So when I didn't get it, I asked for help a lot, I asked for help from people, mainly the people that I knew, or then the classmates that I had known back then. People always... like this... 'Yes, I can help you', some person like... so people explained or then I asked to log in to Discord to help, and I researched quite. So... I was going to bed a little later, but I tried to search for, I tried to study again to understand better what I hadn't been able to get before\footnote{See the original excerpt in Brazilian Portuguese in in Appendix Section \ref{interview-exc-ss:chavo-iq13}.}”.
\end{quote}

Quico structured a linear process starting from basic to complex levels of difficulty:
\begin{quote}
    “So, depending on the project, of what I've to learn, I usually see... so... it is the track that you must follow. [...] So, going from you begin... intermediate and more advanced level. At this moment, I am looking to do in this way to have a performance there, a better development, do you understand? A better flux of development. So, this is the approach that I am looking for. So, I get to use this more for programming. I don't know if [this works] for the other courses, but in programming, I do in this way. When I begin to learn something, I begin to see the basics there, that, usually, every programming language has always the basics that you must learn. So I'm gonna get for other things, more difficult things, and, in this way, I am going scaling, do you understand?\footnote{See the original excerpt in Brazilian Portuguese in Appendix Section \ref{interview-exc-ss:quico-iq6}.}”.    
\end{quote}
He highlighted that during these stages he usually alternates the study place to avoid boredom and guarantee a better learning disposition:
\begin{quote}
    “So... as I spend a lot of time at home in the morning and afternoon period... It's more in this way in my bedroom, reserved there, and on the computer, studying. So, I don't know if this is related to the question of learning by myself, but when you are there solely, sometimes it gets dull, if you do that every day, do you understand? It's the same thing... it gets tedious. So, sometimes, I look for... hmm... I wake up in the morning and don't go to the computer to see something. I stay in another place of my home, doing nothing else, for... I don't know... unwind the mind, for not always doing that same thing. So... but... of the physical place that I stay in my home, it is in the bedroom.

    And related to the college, it is the GRAD [Computing Lab], the labs that have computers. So, when I need to study anything, so... not now, but at the beginning... at first and second terms, I went a lot there, when I arrived [at college]. I arrived, I went to GRAD, I went to do something that I would need to do of programming, of the courses. So, this was very at the beginning. Or, when I'm at my home, the physical place is more my bedroom, and when I'm here [university], it is the GRAD\footnote{See the original excerpt in Brazilian Portuguese in Appendix Section \ref{interview-exc-ss:quico-iq11}.}”.
\end{quote}

\subsection{People as Resource}
\label{results-ss:people}

Concerning interacting with people to promote their \gls{SDL}, Chavo prefers to contact friends or closer people if necessary. When he can not solve your information need, he recurs to other classmates, but only afterward: 
\begin{quote}
    "I try to get in touch with my friends. So... those who I live more together generally. So, usually, I ask them: `People, did you understand what the professor asked?' or `Did you understand that subject?'. Because we have our group, so I ask them.

    I think they are more or less this way... 7 people, more or less 8 people. So I try to speak to the folks. If I cannot, or if I don't understand, I ask another group of all class people who are more familiar from the first term. So I try to ask them. So I ask for help from folks, and that person helps me, and we are gonna understand ourselves. And it's good because if we have another person with a question, so we even help them... happens a Discord [meeting], happens a little call. So, we unfurl to understand the subject. Usually, I do this\footnote{See the original excerpt in Brazilian Portuguese in Appendix Section \ref{interview-exc-ss:chavo-iq8}.}".
\end{quote}
Related to teamwork activities, he thinks about himself as an adaptable person. However, there is a preference for working in groups with more affinity people:
\begin{quote}
    "Yes. I am quiet about working in groups, I can calmly... I can adapt myself. Any group you put me in, so... I can adapt with the people. This is something easy for me. But, so... I like a lot to work with the people I have more affinity with because this feels better, because the people already know more about me, about my routine too, and I also know about their [routines]. So it gets more peaceful because we already know... we know each other better, so... you know... 'Ow, let's do that part, others do another part'. We decide adequately, and it gets better. But if I cannot, I... Any group, so...with the people that put me, I can unfurl it. Mainly with the people of the term that we are together, since the first [one]. I have a very good affinity with everybody, so any group is well\footnote{See the original excerpt in Brazilian Portuguese in Appendix Section \ref{interview-exc-ss:chavo-iq9}.}".
\end{quote}

Quico expressed a similar standing to Chavo concerning interacting with people as much as working in teams. However, he detailed his preference for friends to do teamwork activities:
\begin{quote}
    About negative [experience] is more, so... when someone isn't, so... doing many things and so on. So... usually this happens when... as it's happening, for example now, in the Business course. We are in groups that not everybody we know. Not everybody from the group we know in this case. So, in this scenario of people you don't know and so on, someone who doesn't do something usually tends to be a negative point for me. Do you understand? 

    Because, so... it's normal. But so when there is a group that everybody knows themselves, for me, so... someone isn't doing many things there. As everybody knows themselves, this doesn't become a concern, do you understand? Being there among ourselves, among friends... so this doesn't become a concern.

    \colorbox{black!15}{Me: There is empathy, too.}

    Yes, empathy... we sometimes understand what is happening due to something... [someone] doesn't know... [someone] has many problems. So, we end by revealing... I end by revealing this. So, from the negative viewpoint this is the question of a scenario that I don't know everybody. Do you understand?

    And positive [experience] is exactly this... of you being there among friends and so on, and the work flows and... I don't know. That's all. The work flows. Being among friends there... it's the positive that I see, so... in a group. Do you understand? It's more for this side\footnote{See the original excerpt in Brazilian Portuguese in Appendix Section \ref{interview-exc-ss:quico-iq9}.}".
\end{quote}
The first reason is the better comprehension of friends concerning extra class demands (e.g., home tasks, leisure). Empathy is more accentuated than no-near classmates, allowing a better flux of activities among them and an excellent environment to work. And the second reason is a consequence of the first one. Due to the lack of empathy, he understands that there is a lack of commitment to the group, disturbing the ongoing activities.

It is essential to mention that Quico initially answered this part of the interview differently:
\begin{quote}
    "But... I try to do it by myself. Usually, I try to do it by myself for... Because, so... it's something that I'm seeing now in... when I'm working... that we are doing these projects and we have the teams we are participating in and so on. So, [this is] one thing that I see a lot. Sometimes, people do something, a part of the project, and I didn't do it. And as I didn't do it, so I don't know that part he did, do you understand? So I tend to do it alone as long as possible because I have this knowledge, do you know?

    Because, so... if you don't do it, you usually don't have it. The thing needs to be practical... you must do it, so it's impossible to escape. So I try to do it by myself and don't ask other people to do that. Do you understand?

    But so it arises that... if I cannot, so I speak to a person, talk, and so on. As it's a group question and so on, I tend to do that to understand better, but I also try to involve people to do that together, do you know?\footnote{See the original excerpt in Brazilian Portuguese in Appendix Section \ref{interview-exc-ss:quico-iq8}.}".    
\end{quote}
He shows his preference to do activities alone and usually checks the right progress with the requesting person.

\subsection{Non-Human Resources}
\label{results-ss:non-human}

Regarding interaction with non-human resources, Chavo prefers digital resources (e.g., websites, YouTube) to physical ones:
\begin{quote}
    “When there is something that I don't really know what I'm seeing, I first search for it on the Internet, and I try to look deeper to see if there is documentation. I like a lot to see documentation or search for videos on YouTube. And there also are many things. There are many good materials on the Internet. So, I first focus on these two things. So I try to find out and understand how it works\footnote{See the original excerpt in Brazilian Portuguese in Appendix Section \ref{interview-exc-ss:chavo-iq6}.}".
\end{quote}
He mentioned the use of a classical algorithms book that he borrowed from the university library:
\begin{quote}
    "I've already used the physical library a lot when looking for the algorithm book. Why the algorithm book? As there are many things on the Internet, but some specific things you cannot locate on the Internet or, so... you only find if you research in English, for example. And you need to search for a lot to get to find. And there are other things that are in the book, so I guessed more easily. So... there are reference books on algorithms and data structures... I think it has a thousand pages. I cannot remember the author's name now, but I know it has everything. So I used to get it. I used to check if the CCEN [Exact and Nature Sciences Center] library had it. If I hadn't, I downloaded or... because the book is kinda expensive. So I used to unfurl myself. 

    But I like to use the book when it's a little more difficult to find on the Internet or when there is some specific thing that the professor asks: `You will not go to find it on the Internet', or it will be more difficult to get. So I like to get a reference book here... this facilitates more the things for me\footnote{See the original excerpt in Brazilian Portuguese in Appendix Section \ref{interview-exc-ss:chavo-iq10}.}".
\end{quote}
However, the impression that he recurred from the physical book because he has not had access to the digital version yet.

Quico, similar to Chavo, also prefers digital resources:
\begin{quote}
    "Today, what I can remember are those that I said... it's more YouTube and ChatGPT sometimes, but I cannot think of another thing I use to learn and so on. I think that's it. It's more toward for... when I don't know on YouTube, I go to Google because, sometimes, I read to understand [something] it's better than listening to someone tell me, do you know?

    And when I'm reading and don't understand, so it's the inverse. I go to YouTube to understand someone speaking... it's the best. So that's it. This is the scope that I'm using in methodology to learn and so on... that's it. I'm in this world and so on\footnote{See the original excerpt in Brazilian Portuguese in Appendix Section \ref{interview-exc-ss:quico-iq10}.}".
\end{quote}
When talking about them, he mentioned how he uses ChatGPT as a resource. First, he looks for information in classic digital sources (e.g., papers). Once unsuccessful, he asks ChatGPT about the subject. When satisfied with the answers, he checks the generated answers to other digital sources through Google searches, verifying their consistency (like a triangulation process):
\begin{quote}
    "I'm understanding. So... I think what I ask more is [about] things that I need to answer from assignments and so on. So that's it. Let's suppose this assignment here that I had to do from \gls{BPM}. I had to do a report and had topics there for me to develop the report. So what did I do? As she wanted that we used papers, I was searching for papers, and it wasn't coming the thing I wanted. So what did I do? I put it there and asked it to develop a paragraph related to a topic that was there. Great. So I did this, I read it there, and so I... 'Ok, interesting'. I captured what I wanted from there and went to see... no paper. I went to Google... normal. So I saw there that the things matched themselves and so on, do you understand?

    So, sometimes, what do I do? I have a question there. Sometimes, I cannot find what I wanted on Google, [then] I put there on ChatGPT, ask it, [and] it sheds light. So, good. I read there, I gonna see again on Google to check if it has relation, do you know? Because it can be that what it's saying it's not true, so it's not related. So that's it. I ask there what I need. I check there, and I do a fast scanning to complement, do you understand?\footnote{See the original excerpt in Brazilian Portuguese in Appendix Section \ref{interview-exc-ss:quico-iq10}.}".
\end{quote}

\subsection{Place as Resource}
\label{results-ss:place}

About places as a resource, Chavo prefers to self-direct learn at home. Studying at work and university is a good choice but as a second option:
\begin{quote}
    "Well... I think that there are two places that I like to study more, in my bedroom mainly. It's because I usually stay more [time] here, so as I stay at my home for the most part alone, I've got used to it and no problem. But when I must go to college or, for example... or at work, in the office... it's possible to work quietly, review the class quietly, but at the college mainly.

    I'm doing this a lot this week. I'm gonna have this tomorrow... tomorrow will be a rush. And near us, from \gls{MIS} [\acrlong{MIS} classroom] there, that is aside literally, there are some little benchs there, there are outlets. So... I like there because it's ventilated and, even with some people, the people respect the silence. It's calm to study there or at the library. But at the library, as it gets a little far, I prefer to get closer to \gls{CIn} due to Wi-Fi\footnote{See the original excerpt in Brazilian Portuguese in Appendix Section \ref{interview-exc-ss:chavo-iq11}.}".    
\end{quote}
Although he likes to be at home, he correlates learning at home to more chances to procrastinate activities due to the temptation to play video games or watch streaming content, for instance. He associates studying at university with focus:
\begin{quote}
    "[...] Because at home, you procrastinate a little. This happens to me relatively always. So... at college, I can have more focus really. I can concentrate and stay focused for more time. At home, I have some distractions, but I'm working to try to improve. [...] At college, when I'm alone, for example, in a place studying... I'm a little more focused for more time than at my home because there is, for example... WhatsApp and people messages... many things. So I'm at home. It's just to stand up, go there and come back. But beyond this... I think that's this\footnote{See the original excerpt in Brazilian Portuguese in Appendix Section \ref{interview-exc-ss:chavo-iq11}.}".
\end{quote}

Quico also prefers to study at home, specifically in his bedroom. He also uses the university laboratories to study as an alternative choice (see second Quico's quotation in Section \ref{results-ss:strategy}).

\subsection{Time as Resource}
\label{results-ss:time}

Chavo has an additional factor concerning time because he works. His job occupies 30 hours per week, ranging from morning or afternoon shifts. The night shift is reserved for his undergraduate program:
\begin{quote}
    So... now I work. I wake up at 6 am, about 6:20 am. [...] I rest a little, study, have breakfast. So between 6 and 10 am. 10 am I begin to work. So I stop at noon, come back at 1 pm. So, from 1 pm to 4 pm, it's the traineeship that I got with the Federal [University], that is 6 hours.

    [...] So generally, that's it. I have classes at college. When I come back, I get on the bus to CCEN [Exact and Natural Sciences Center], and I go to my home. So... I review the things, and I go to bed. Eating and sleeping\footnote{See the original excerpt in Brazilian Portuguese in Appendix Section \ref{interview-exc-ss:chavo-iq5}.}".    
\end{quote}
He shared that this job is not a guarantee for his studies but contributes significantly to fund his public transportation and feeding in the university. He continued saying that could quit his work but the familiar budget would get with tight:
\begin{quote}
    "I got it. I think not... due to study... I can... I can stay with study and no work because it's [a] Federal [university], but it's a bit more complicated. Because... as it's only me and mom at home... and before my job, it was more difficult because of the bus fares and food at college. And this... because... wanting or not, bus fares cost a lot and food too. But so... I think that's it. But before... Wait, I think I... Wait a minute. Can you repeat the question, please? Because I got lost.

    \colorbox{black!15}{Me: I can. I'm asking the following: is your job essential to guarantee your studies?}

    Right. Ok... It would be more related to transportation and a little to food. Because, so... it's possible to stay [without working]... it's possible. You have to be well-tight. For example, before I began my traineeship, what did I do? I used to do it this way... as going to college is peaceful, with the sun still and so on... all right. I go walking many times because it's near, so it's already saved one bus fare. I would only go [from bus] in the coming back. [...] It's possible to unfurl yourself without, but it gets more complicated, do you know?\footnote{See the original excerpt in Brazilian Portuguese in Appendix Section \ref{interview-exc-ss:chavo-iq12}.}".
\end{quote}

Concerning Quico's context, his major occupation is the undergraduate program: 
\begin{quote}
    "So... that's it. It's not clear in my mind because I don't have any experience in this area yet, [neither] I'm not looking for a traineeship. So maybe when I look for a traineeship and begin, it will be clearer, do you know?\footnote{See the original excerpt in Brazilian Portuguese in Appendix Section \ref{interview-exc-ss:quico-iq3}.}"

    [...] I think it's... it's not ruled, but it's always the same thing. As I said to you, in the morning, I have... from 10 am to 2 pm, I'm free... Not free! I'm at home, but doing university things, maybe household things.

    So the day boils down to this when it's normal... when it has classes. From 10 am to 2 pm, I'm doing some university things. So when it has come close to 3 pm, I get dressed to go out.\footnote{See the original excerpt in Brazilian Portuguese in Appendix Section \ref{interview-exc-ss:quico-iq5}.}".
\end{quote}
He does not work and has a facility to go from private car to university:
\begin{quote}
    \colorbox{black!15}{Me: How do you arrive at home? Do you go by bus?}

    I go by car. I live in Tangamandapio\footnote{Santiago Tangamadapio is a little city in Mexico southwest (\url{https://en.wikipedia.org/wiki/Tangamandapio}). It is referred to in Chespirito as the city of the postman Jaimito. I preferred not to reveal the real city of Quico in this transcript for research ethical reasons.} When the highway is not jammed, it's nearly 22 minutes\footnote{See the original excerpt in Brazilian Portuguese in Appendix Section \ref{interview-exc-ss:quico-iq5}.}".
\end{quote}
The familiar income does not seem an imminent concern in his answer about this topic. Although Quico lives so far from university, the time spent by him to come to it is practically the same as that spent by Chavo (who lives near to university but uses public transportation).

\subsection{Evaluation}
\label{results-ss:evaluation}

Related to \gls{SDL} evaluation, Chavo checks if he is “in the right way,” mainly by means of people. He mentioned that he checks with friends, professors, and monitors if his learning route is appropriate. He also creates a ‘to-do list’ composed of topics to learn, and each topic has a deadline. Thus, he recurs both human feedback and this personal activities list.

Quico also checks his progress with people, but mainly with his parents and, secondarily, with his friends. He also verifies his learning route via levels of progression, assessing if he is at the beginning, middle, or end of his learning journey. At last, he asserted that another essential factor is if his efforts are contributing to his “labor market absorption”, pointing out his learning intentionality focused on professional formation.
