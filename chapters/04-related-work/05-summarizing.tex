\section{Summarizing and Situating Results}
\label{rel-work:sum-results}

In this section, I present the summary of the results, positioning my research before the broader area (Section \ref{rel-work-sum-ss:summary}) and some of the mapping threats (Section \ref{rel-work-sum-ss:threats}).

\subsection{Summary and position}
\label{rel-work-sum-ss:summary}

\subsubsection{Research (DSRQ.1)}

About \gls{DSRQ}.1, most of the papers are primary works (29). There is only (i) one secondary paper \cite{lai:2023}  that reviews the broader area from the \gls{SRQ} perspective and (ii) one essay \cite{michaelis:2022} that is usually in frontiers between primary and secondary categories. There is a reasonable balance between quantitative (10), qualitative (8), and mixed-methods (10) approaches. It was not possible to identify the research methodology in two papers \cite{michaelis:2022,akalin:2021} and the secondary work used the \citeonline{kitchenham:2007} guidelines.

I chose a primary and qualitative approach for this research. This option reflects the research that looks for a better understanding of real scenarios when inequalities of opportunity can arise. I conducted a basic qualitative research using quantitative data to support triangulations and sampling choices.

\subsubsection{Context (DSRQ.2)}

In relation to \gls{DSRQ}.2, the papers are balanced into higher (14, including graduate studies) and basic education (13, including high school and professional formation). The possible reason for this equilibrium is the inclusion of computing in basic education in many countries. The work of \citeonline{arawjo:2021} is an example of an exception, focusing on informal education (3) too. Most papers investigate the research context in \gls{USA} (16), followed by Asia (6), and Europe (5). Africa \cite{arawjo:2021}, Latin America \cite{roque-hernandez:2021}, and Oceania \cite{shahin:2022} have only one work each.

I investigated \gls{CSE} context in this thesis. However, the Brazilian scenario brings a difference when focusing on developing countries. Only seven papers (of 31) have their contexts situated in the Global South\footnote{I used the demarcation criteria adopted by \gls{BISA}. See more in \url{https://www.bisa.ac.uk/become-a-member/global-south-countries}.}. Only one of them investigates Latin American contexts \cite{roque-hernandez:2021}, for instance. There is a need for more research in developing countries into this cut.

\subsubsection{Equity (DSRQ.3)}

About \gls{DSRQ}.3, the papers approach a wide range of equity issues prevailing gender (13), performance (10, including self-efficacy and expertise), and race (7, including culture and nationality) issues. Few works investigated sense of belonging (2), participation (2), and access (1) issues. In relation to a general equity theory (or framework), no work uses a consolidated theory/framework, usually building its theoretical background from various constructs spread over several references. In this perspective, four works drew my attention concerning their theoretical background, highlighting  intercultural computing \cite{arawjo:2021} that resulted from a specification of a previous theory (intercultural learning). Other three works refer to epistemic injustice \cite{love:2021}, gender gap \cite{bodaker:2023}, and interest development theory \cite{michaelis:2022}.

This research used the \gls{SES} to help to choose participants during data collection. However, the thesis's uniqueness resided in using a general equity theory / framework, allowing me to investigate a Brazilian context under well-informed and general equity constructs. A crucial characteristic of my research is using \gls{CA} as an equity framework. The richness of this choice increases when we consider the singular reality faced by developing countries, deepening the discussion of the deprivation of freedoms and not only about the presence/lack of resources. 

\subsubsection{Active Approach (DSRQ.4)}

At last, in relation to \gls{DSRQ}.4, most of the papers investigated collaborative learning (25), being pair programming the major part. Few works adopted other approaches like \gls{PBL} (2), peer-mentoring (2), mixed-approaches (2), and project-based learning (1).

I looked into the \gls{SDL} as an active approach. The potentiality of my research was focusing on a more general approach (\gls{SDL}) than a specific one (e.g., andragogy, \gls{PBL}). Although I investigated a Brazilian context with a \gls{PBL} scenario,  the understanding of \gls{SDL} is crucial because it dialogues and is part of several other active approaches (see Section \ref{sdl-sec:relations}).

\subsection{Mapping Threats}
\label{rel-work-sum-ss:threats}

I list three main threats of this scoping study. First, I conducted only the first two iterations of snowballing until now. It is usually necessary to finish this kind of review in three to five iterations. Although the first two iterations represent a good sample of related work, the new start set is composed of 20 papers and can cover more potential and relevant works.

Second, I used Scopus base as my source of citations during the forward snowballing. The snowball effect is sensitive to the search database, and it is possible that the engine did not find any work.

Lastly, my seed start set is composed of one paper only \cite{lewis:2015}. It is a well-known fact that the start set diversity can bias the search graph of citations and references, leading to undesirable "local minima and maxima". This start set may even "burst the bubble" of a certain kind of papers, but there is no guarantee that any relevant paper is unachievable.
