\section{Charting Data}
\label{rel-work:charting-data}

I created a list of essential items of information that should be obtained from selected papers from the first iterations. This list helped me to arrange the works and situate how the area is being explored and which potential gaps and challenges should be addressed. Each item has a related \gls{DSRQ}, contributing to answer \gls{SRQ}. Table \ref{tbl:papers-chart} details this list.

\begin{table}[htb]
\caption{List of the papers of the new start set identified during the 
Iteration 1 after snowballing strategy.}
\label{tbl:iteration-1-list-papers}
\centering
\rowcolors{1}{}{lightgray}
\begin{tabular}{
    m{7cm}|
    m{7cm}
}
    \hline
    \multicolumn{2}{c}{
        \textbf{Old Start Set}
    }\\
    \hline
    \citeonline{arawjo:2021} &
    \citeonline{ayub:2020} \\

    \citeonline{bodaker:2023} &
    \citeonline{grabl:2024} \\
    
    \citeonline{gransbury:2022} &    
    \citeonline{izhikevich:2022} \\
    
    \citeonline{love:2021} &    
    \citeonline{lui:2020} \\
    
    \citeonline{lyttle:2020} &    
    \citeonline{musaeus:2022} \\
    
    \citeonline{ying:2021} &
    \\
    \hline    
    \multicolumn{2}{c}{
        \textbf{New Start Set}
    }\\
    \hline
    
    \citeonline{akalin:2021} &
    \citeonline{alvarado:2022} \\
    \citeonline{bowman:2020} &
    \citeonline{broll:2021} \\
    \citeonline{demir:2021} &
    \citeonline{eglash:2020} \\
    \citeonline{goode:2021} &
    \citeonline{kung:2022} \\
    \citeonline{lai:2023} &
    \citeonline{lott:2021} \\
    \citeonline{michaelis:2022} &
    \citeonline{nakai:2023} \\
    \citeonline{roque-hernandez:2021} &
    \citeonline{shahin:2022} \\
    \citeonline{su:2023} &
    \citeonline{tan:2024} \\
    \citeonline{toro:2024} &
    \citeonline{tseng:2024} \\
    \citeonline{wei:2021} &
    \citeonline{ying:2021b} \\

    \hline
    
\end{tabular}

  \par\medskip\ABNTEXfontereduzida\selectfont\textbf{Source:} Created by the author (2024). \par\medskip
\end{table}

\begin{table}[htb]
\caption{List of items of information obtained from the selected papers during the first iteration. Each item has a related derived secondary research question.}
\label{tbl:papers-chart}
\centering
\rowcolors{1}{}{lightgray}
\begin{tabular}{
    m{2.5cm}|
    m{7cm}|
    m{3cm}
}
    \hline
    \multicolumn{2}{c}{Item of Information} &
    Related \gls{DSRQ}\\
    \hline
    Research &
    Type & \gls{DSRQ}.1 \\
    & Kind & \gls{DSRQ}.1 \\
    & Methodology & \gls{DSRQ}.1 \\
    \hline
    Context &
    Educational Level & \gls{DSRQ}.2 \\
    & Country / Region & \gls{DSRQ}.2 \\
    \hline
    Equity & 
    Equity Issue & \gls{DSRQ}.3 \\
    & General Equity Theory / Framework & \gls{DSRQ}.3 \\
    \hline
    - &
    Active Learning Approach & \gls{DSRQ}.4\\
    \hline
    
\end{tabular}

\par\medskip\ABNTEXfontereduzida\selectfont\textbf{Source:} Created by the author (2024). \par\medskip

\end{table}

The filled data chart from 31 papers is available in Appendix \ref{chap:data-charting}. I used this charting not only to answer \gls{SRQ} but to situate \gls{MRQ} in the broader area, as will be seen in the next section.