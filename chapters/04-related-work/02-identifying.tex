\section{Identifying the Research Questions}
\label{rel-work:res-questions}

My \acrfull{MRQ} is “how do \acrfull{CSE} students conduct their \acrfull{SDL} in developing countries from the lens of the \acrfull{CA}?” (stated before in Chapter \ref{chap:intro}). However, aiming to situate my project into a broad network of the research area, I use a \gls{SRQ}: “which and how are the works in \acrfull{CEd} involving equity issues and active learning?”. In order to address the research question we consider four \glspl{DSRQ}:
\begin{itemize}
    \item What are research methodologies (or kind of review, being a secondary work) used? (\gls{DSRQ}.1);
    \item What are the contexts (country, education level) involved? (\gls{DSRQ}.2);
    \item What are the equity issues investigated? (\gls{DSRQ}.3); and
    \item What are active approaches adopted? (\gls{DSRQ}.4).
\end{itemize}

When I refer to active learning, I use the general definition of \citeonline{bonwell:1991}, which assert that it is “anything that involves students in doing things and thinking about the things they are doing” \cite[p.~19]{bonwell:1991}. The authors mention that there is no precise definition of active learning (as also \citeonline[p.~223]{prince:2004}), but some general characteristics can be listed as:
\begin{itemize}
    \item “Students are involved in more than listening;
    \item Less emphasis is placed on transmitting information and more on developing students’ skills;
    \item Students are involved in higher-order thinking (analysis, synthesis, evaluation);
    \item Students are engaged in activities (e.g., reading, discussing, writing);
    \item Greater emphasis is placed on students’ exploration of their own attitudes and values” \cite[p.~19]{bonwell:1991}.
\end{itemize}

