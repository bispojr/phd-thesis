\chapter{Related Work}
\label{chap:rel-work}

 There are various reasons to structure a related work chapter \cite[p.~14]{booth:2016-slr}. For this thesis, I am interested in (i) locating my work within the existing literature and (ii) justifying its originality. Doing this in a qualitative investigation requires me to establish some considerations.
 
Although the expression “on the shoulders of giants” can represent a humble stance before the complexity and greatness of produced knowledge through scientific endeavor, it is possible to interpret it in a positivist way. The discussion proposed here does not intend to build one brick more in a cartesian wall of science. I propose to establish one more link to the big network of produced scientific knowledge. In this network, each node links a perception of reality historically, through dialogues with the contribution of colleagues situated in space and time.

In this perspective, it is necessary to move away from a kind of technicism that tries to specify and discretize every step of a given methodology. A hidden pitfall is to use this agenda to describe reality, just as it is, in a positivist way. What we need is a rigorous and, if possible, systematic description to explicit the main aspects of our methodological approach. A systematic approach in a qualitative research is not strictly a matter of reproducibility, like we try to replicate an experiment. However, it is a concern of rigor that legitimizes the quality of our research and allows other researchers to structure a possible transferability (see more in Chapter \ref{chap:res-methodology}).

One of these pitfalls is the own systematization process. There is a particular fear among qualitative researchers that the emergence of findings during and after the data collection, for instance, can be harmed due to the process inflexibility. About it, \citeauthoronline{meinefeld:2004} points out:
\begin{citacao}
    “[...] this does not mean that the result has been predetermined, as critics sometimes claim: it is only the framework of the dimensions involved in the investigation that has been fixed, but not their concrete manifestations of content.” \cite[p.~157]{meinefeld:2004}.
\end{citacao}

Understanding and admitting that not every systematic approach contributes negatively to rigorous and well-conducted qualitative research is necessary. A systematic approach can provide a solid foundation to help other researchers adequately posit their investigation into the major network of produced knowledge. In this thesis, I use a systematic mapping instead of a systematic review. \citeonline[p.~1]{petersen:2015} establish the distinction between them:
\begin{citacao}
    “While systematic reviews aim at synthesizing evidence, also considering the strength of evidence, systematic maps are primarily concerned with structuring a research area”.    
\end{citacao}

Systematic reviews tend to be more committed to the movement of evidence-based science, quantifying and discretizing this process overly. Systematic maps pervade several research perspectives and allow to structure a related work to qualitative research without leaving essential aspects in this approach.

In summary, I can retake the aims of this systematic mapping and assert that (i) \underline{locating my} \underline{work within the existing literature} is important not only for situating this research but also to help other researchers to situate their further works, and (ii) \underline{justifying its originality} in a qualitative approach is, above all, evidencing the need to uncover the implicit meaning in a concrete context. 

The remainder of the chapter is organized as follows. Section \ref{rel-work:scoping-snowball} presents the scoping study aligned with the snowballing strategy. Important elements of a systematic map such as research questions (Section \ref{rel-work:res-questions}), snowballing start set (Section \ref{rel-work-ss:start-set}), and inclusion and exclusion criteria (Section \ref{rel-work-ss:inc-exc-criteria}) are detailed. Section \ref{rel-work-ss:first-iterations} describes the first two snowballing iterations, showing both backward and forward steps. At last, after charting data (Section \ref{rel-work:charting-data}), Section \ref{rel-work:sum-results} exposes the data analysis of these iterations, situating this thesis into a general frame of the research area.
