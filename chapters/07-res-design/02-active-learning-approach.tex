\section{Active Learning Context}
\label{res-des-sec:active-learning}

As presented before (Section \ref{sdl-sec:relations}), \gls{SDL} establishes relations with many active learning approaches. In this section, I present the active learning context in which the \gls{SDL} construct was investigated in this research. I detail the \gls{PBL} adopted in this research context (Section \ref{res-des-sec:context}), delineating the \gls{PBL} By-Cycles Framework \cite{alexandre:2018} from four essential steps in this evolution journey (in order of arising): (i) \gls{PBL}-Test, (ii) \gls{xPBL}, (iii) \gls{PBL} Framework, and (iv) \gls{PBL-SEE}.

The first important step of \gls{NEXT} aiming to structure the learning processes in \gls{PBL} for \gls{CEd} was the \gls{PBL}-Test \cite{santos:2013}. \gls{PBL}-Test is a model to evaluate the maturity of
teaching processes in a \gls{PBL} approach. The idea is to verify the perception of all \gls{PBL} stakeholders (e.g., facilitators, tutors, students) concerning \gls{PBL} principles (as presented in Section \ref{sdl-relations-ss:pbl}). One of the main results after the \gls{PBL}-Test application is to locate what level of maturity your \gls{PBL} is (that can be: insufficient, initial, satisfactory, good, or excellent level).

The second step in the \gls{NEXT} evolution was the proposal of the \gls{xPBL} \cite{santos:2014}. \gls{xPBL} is a methodology for managing \gls{PBL} when teaching Computing. The idea is to provide an alternative to yPBL methodology \cite{exposito:2010}, providing a relationship between \gls{PBL} principles and five methodology elements (obtained from previous \gls{NEXT} research experiences). These key methodology elements are (i) problem, (ii) environment, (iii) content, (iv) human capital, and (v) process. The authors presented, for each element, a pathway to conduct a 5W2H technique \cite{klock:2016} aiming to help computing educators in a \gls{PBL} course design.

The third \gls{NEXT} step was the proposal of the \gls{PBL} Framework \cite{rodrigues:2016}. The framework idea is to ensure satisfactory results by using \gls{PBL} in \gls{CEd}, reusing as a base the Deming cycle: \gls{PDCA} \cite{dudin:2015}. \gls{PBL} Framework incorporates both \gls{xPBL} (Plan Phase) and \gls{PBL}-Test (Act Phase), also signaling an authentic assessment as one of its key components (Check Phase).

The last step in this evolution was an authentic assessment model for
\gls{PBL}-Based Software Engineering Education: \gls{PBL-SEE} \cite{santos:2016}. \gls{PBL-SEE} address the Check Phase of \gls{PBL} Framework with a structured model, being composed of three levels (i) student assessment, (ii) \gls{PBL} evaluation, and (iii) teaching assessment. The objective of this model is to indicate assessment strategies that guarantee the effectiveness of the \gls{PBL} approach throughout its management cycle. Educational Objectives are established based on \gls{RBT}, associating each verb in \gls{RBT} six levels to \gls{xPBL} five elements. 

Figure \ref{fig:pbl-by-cycles} presents the schema of \gls{PBL} By-Cycles Framework with the main steps detailed here.