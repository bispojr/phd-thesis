\section{Reflexivity in Qualitative Research}
\label{reflexivity:qualitative-research}

Reflexivity arises from some requisites for good qualitative research. \citeonline[p.~14]{merriam:2016-whatIs} assert that:
\begin{citacao}
    “Getting started on a research project begins with examining your own orientation to basic tenets about the nature of reality, the purpose of doing research, and the type of knowledge to be produced through your efforts. Which orientation is the best fit with your views? Which is the best fit for answering the question you have in mind?”.
\end{citacao}
All these recommendations are important for any research. But they are crucial in a qualitative approach because the researcher is the primary instrument of data collection and analysis (see Chapter \ref{chap:res-methodology}). \citeonline[p.~16]{merriam:2016-whatIs} still assert in this direction that:
\begin{citacao}
    “However, the human instrument has shortcomings and biases that can have an impact on the study. Further, there is a particular theoretical framework or lens that informs a research study that the researcher makes visible. Rather than trying to eliminate these biases or ‘subjectivities’, it is important to identify them and monitor them in relation to the theoretical framework and in light of the researcher’s own interests, to make clear how they may be shaping the collection and interpretation of data”.    
\end{citacao}
Thus it matters to clearly express the assumptions and worldview of the researcher, aiming to increase rigor in qualitative research.

However, it is essential to situate what would be a reflexivity activity. Because, at the same time that there is strength when we provide a clear and honest research report, it is also possible we overly deviate from the main focus of investigating the phenomenon correctly. To avoid some pitfalls, we should establish a more delimited structure of the reflexivity practice.

\subsection{What should it be?}

There are no single accepted definitions of reflexivity, but there are promising directions. I adopted in this work the definition of \citeonline[p.~814]{probst:2014} which asserts that:
\begin{citacao}
    “Reflexivity is generally understood as awareness of the influence the researcher has on what is being studied and, simultaneously, of how the research process affects the researcher. It is both a state of mind and a set of actions, both concept and practice”.
\end{citacao}
Understanding the intersubjective dynamics contributes to improving the trustworthiness of research, mainly in the qualitative approach. A significant aspect of this task is related to increasing the research rigor.

Although reflexivity goes through a self-reflection activity, it is necessary to differentiate from it, expanding our understanding. \citeonline[p.~816]{probst:2014}  use the double-arrow metaphor, indicating the two movements existing in this process, both inward and outward viewpoints:
\begin{citacao}
    “Although similar, reflexivity can be differentiated from self-reflection. Self-reflection, or the conscious observation of one’s inner world, is a valued aspect of many disciplines such as psychoanalysis and various forms of spiritual practice. However, it represents the arrow pointed solely, or primarily, at the self, while reflexivity is the reciprocal interplay between the ‘archer’s’ inward and outward viewpoints”.
\end{citacao}

The record of reflexivity activities can be done using multiple means. It is possible to write diary logs, record regular audio notes, take pictures, or even make a video of critical phases of your research. But beyond the simple recording, understanding the difference between reflection-on-action and reflection-in-action (during the research process) is essential. It is vital to conduct the reflexivity process as “an ongoing reflection-in-action throughout the entire research endeavor rather than retrospective reflection-on-action at its conclusion” \cite[p.~815]{probst:2014}. However, this is very hard due to the absence of a scientific culture, which makes it a part of the research project and not a mere activity report.

\subsection{What should it not be?}

When we conduct a reflexivity process, one of the risks is to concentrate the focus on ourselves overly. This process aims to reveal new crucial information for readers (who will access your report) and yourself (who will conduct your research). It becomes meaningless if reflexivity tends to be a narcissistic look only.

\citeonline[p.~813]{probst:2014} assert that “the mechanism of reflexivity may not lie in the specific activity but in the attitude with which it is carried out”. It is necessary to conduct reflexivity tasks not only to seek more rigor. The unbridled drive for rigor can lead to the road of objectification (or a straightforward operationalization). \citeonline[p.~826]{probst:2014} still assert that:
\begin{citacao}
    “Mixing epistemologies by trying to objectify a process that is fundamentally subjective will not, in the end, enhance rigor. The aim, after all, is not the formulaic or confident use of ‘reflective tools’, but engagement in the complex and slippery process of struggling to understand the meaning of human experience”.
\end{citacao}

Another risk is to transform the reflexivity struggle into an intimidatory activity. \cite[p.~212]{hsiung:2008} highlights the risks when we conduct it among other researchers:
\begin{citacao}
    “Because doing reflexivity [...] poses a number of challenges. Students often feel personally threatened by, and are resistant to, the prospect of critically examining their own positions and experiences [...]. Unless students are actively encouraged to be reflexive, they are unlikely to welcome the vulnerability of admitting to errors or imperfections that reflexivity requires”.
\end{citacao}
It is necessary to look for a balance between the exposition of our inwardness and the preservation of our emotional feelings that can be triggered during reflexivity tasks.
