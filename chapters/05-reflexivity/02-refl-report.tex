\section{Reflexivity Report}
\label{reflexivity:report}

I adopted the modalities of \citeonline[pp.~513-514]{longhofer:2012} to do this reflexivity process. These modalities are (i) personal, (ii) ontological, (iii) epistemological, (iv) methodological, (v) theoretical, (vi) normative, and (vii) representational. I describe the first three ones in more detail in the next sections.

This reflexivity report is situated historically. It reflects a self-understanding of the whole period of my Ph.D. studies under advisory by two expert researchers. I am researching computing education at the Federal University of Pernambuco. My research object is to investigate how \gls{CSE} students conduct their \gls{SDL} in developing countries from the lens of capabilities approach \cite{sen:1992,robeyns:2023}.

\subsection{Personal modality}

\subsubsection{Beliefs}
\label{reflex-sss:beliefs}

I am a Christian from an evangelical tradition. Therefore, my worldview is strongly affected by my religious beliefs. As I believe God exists, I am a spiritualist (instead of a materialist). As I believe in a Creator God, I believe in the existence of a single reality (instead of multiple ones). 

I believe the human condition limits the capacities of human beings to know this single reality as it is. Although a single reality exists, human beings cannot perceive it in an appropriate way. Hence, a single reality exists but there are multiple interpretations about it (and not multiple realities). 

Humankind is not ‘good’ (like Rousseau asserts). Human beings have a limited nature. They are limited, fallible, and finite. But there is something of God in all human beings. All human beings carry in themselves “the God-likeness”. In this way, it is possible to capture some aspects of reality, although it is so difficult to share these appropriations with other human beings.

Because of the existence of multiple interpretations, everyone is naturally led to dialog with other people that have different worldviews. It forced me to situate my beliefs into a broader perspective. An essential question for me is: “How do I conciliate my beliefs with my social condition, my political commitment, my life history, and so on”?

\subsubsection{Political commitment}

I used to be closer to progressive perspectives on the political spectrum. I consider myself a social democrat and, for this reason, a reformist (not a revolutionary, in a Marxist view \cite{schaff:1973}. I believe that computing educators should adopt a non-conformist stance before society. Although I understand the predisposition of public structures to serve as ideological and reproducer apparatuses of society’s status quo \cite{bourdieu:1989}, I am hopeful in the strength of all stakeholders in the educational environment for reforming the school structure and becoming it more human and less oppressive. My stance reflects Freire's position when he asserts:
\begin{citacao}
    “What is posed to the democratic educator, conscious of the impossibility of education neutrality, is to forge themselves a special knowledge, that never should abandon, knowing that motivates and sustains their struggle: if education cannot do all, it can do something fundamental. If education is not the key to social transformations, it is also not a reproducer of the dominant ideology. What I want to say is that education is not an unbeatable power to serve the society's transformation, although I would, neither is the perpetuation of the ‘status quo’ because the dominant decrees it. The democratic critic cannot think that, from the course that coordinates or seminary that leaders, can transform the country. But they can demonstrate that it is possible to change. And this reinforces in them the importance of their political-pedagogical task” \cite{freire:1996-ped-aut}\footnote{This is my translation of the original excerpt in Portuguese that as follows: \textit{“O que se coloca à educadora ou o educador democrático, consciente da impossibilidade da neutralidade da educação, é forjar em si um saber especial, que jamais deve abandonar, saber que motiva e sustenta sua luta: se a educação não pode tudo, alguma coisa fundamental a educação pode. Se a educação não é a chave das transformações sociais, não é também simplesmente reprodutora da ideologia dominante. O que quero dizer é que a educação nem é uma força imbatível a serviço da transformação da sociedade, porque assim eu queira, nem tampouco é a perpetuação do ‘status quo’ porque o dominante o decrete. O educador e a educadora críticos não podem pensar que, a partir do curso que coordenam ou do seminário que lideram, podem transformar o país. Mas podem demonstrar que é possível mudar. E isto reforça nele ou nela a importância de sua tarefa político-pedagógica”}.}.
\end{citacao}
This position allows me to dialogue with the macro tendencies of liberalism: the competitive and statist capitalism \cite[pp.~84-95]{libaneo:2011}. Competitive capitalism is closely related to conservative liberalism, promoting the free market, efficiency, and quality as values. This macro tendency leads to seeking to reduce the power of the state. On the other hand, state capitalism criticizes conservative liberalism, promoting equality of opportunities as the main idea. This macro tendency leads to seeking to increase the power of the state. 

As a social democrat, I naturally sympathize more with state capitalists than competitive ones. But it is possible to dialogue with both because there are good starting points in the statist liberalism literature that allows me to close my ideas with the liberal thought as a whole.

\subsubsection{Social identity and Research Motivation}

I am self-declared as brown. I was born in a low-income family in the Brazilian northeast. But, before other low-income families, my one had a bit more capabilities, and it was possible to guarantee my siblings and me a great formal education. My father (\textit{in memoriam}) was black and did not finish his undergraduate studies. My mother (\textit{in memoriam}) completed her high school studies only when she was in the adult phase. I am a first-generation undergraduate \cite{ives:2020}.

To understand how my social identities have shaped my research, I will tell a little about my research motivation originating from affirmative policies. I did not know for sure what an affirmative policy was. But one day, I needed it. I was always an effort student and was taught from childhood to "fight by my dreams" without waiting for help, contribution, or any "alms" from people, institutions, or government. But my belief was confronted when it came up against a hard reality. If I did not accept myself as the target of an affirmative policy, I would not begin my undergraduate studies.

This story begins in this way. At the end of my high school studies, I had the opportunity to receive a partial scholarship to attend a preparatory course for military tenders (considered one of the most difficult to apply in Brazil). The “Pré-ITA” was a differentiated course. The idea of wanting to be approved in these tenders is considered too bold still today, having a high competitiveness. I had this course during the whole year of 2004. I had the opportunity to do seven tenders: ITA, IME, EsPCEx, ESA, AFA, UFPE, and UPE. To my sadness (and surprise), I did not get to be classified in any of these tenders. The closest tender I got was in the UFPE tender in Computer Science, where I reached the position of 109º for 100 vacancies. As Computer Science is a well-disputed course, only six of them gave up on entering, and, unfortunately, I did not get to do my undergraduate studies at UFPE.

It was not simple to process this result. I felt incompetent. I reached the point of speaking to myself that I acted wrong in wanting to apply for such well-disputed tenders. However, something unexpected happened to me. I had done the ENEM exam. I got an outstanding grade (including 100 on the essay part). In the same year of 2004, the University for All Program (ProUni) was created. The ProUni granted me a full scholarship for my undergraduate studies at the Catholic University of Pernambuco (UNICAP). This program used criteria like the fact that (i) I have studied the whole high school in a public institution, and (ii) I have a low socioeconomic status (income of 1.5 minimum wage \textit{per capita}).

Although my life story does not end in this cycle, it has had a very “happy end” for me. I did my undergraduate studies at UNICAP, where I was a laureate in my class. My feelings at the end of my undergraduate were so different from my initial ones. I went from a self-depreciation condition to fresh self-esteem.

% \vspace{0.3cm}

% \fbox{
%     \begin{minipage}[htb]{0.9\textwidth}
%         \vspace{0.3cm}
                
%         \colorbox{gray!30}{% create a colored box
%             \makebox[0.975\textwidth][l]{% center the text on the page
%                 \ \ \textbf{Further Writing}
%             }
%         }

%         \vspace{0.1cm}
        
%         \begin{itemize}
%             \item Translating these paragraphs to English:
%             \begin{itemize}
%                 \item Poderia contar muitas histórias aqui das minhas relações com as políticas afirmativas. Mas o que eu quero comunicar com esta carta é a minha identificação real com o meu objeto de pesquisa. Embora já exista no Brasil uma política afirmativa em relação ao ingresso no ensino superior público, muitos são os desafios para garantir que haja igualdade de oportunidades para todos os estudantes enquanto eles cursam a sua graduação. O problema da equidade diante dos desafios da permanência do estudante pode ser enxergado através de várias lentes. E a lente que eu escolhi para investigar é a perspectiva do nível socioeconômico do estudante. Essa lente é importante porque tem uma função crucial dentro do nosso modelo econômico.
                
%             \end{itemize}

%         \end{itemize}

%         \vspace{0.25cm}
        
%     \end{minipage}
% }

% \vspace{0.3cm}

% \fbox{
%     \begin{minipage}[htb]{0.9\textwidth}
%         \vspace{0.3cm}
                
%         \colorbox{gray!30}{% create a colored box
%             \makebox[0.975\textwidth][l]{% center the text on the page
%                 \ \ \textbf{Further Writing}
%             }
%         }

%         \vspace{0.1cm}
        
%         \begin{itemize}
%             \item Translating these paragraphs to English:
%             \begin{itemize}
%                 \item Outro aspecto importante do meu objeto de estudo são as mudanças que estão sendo adotadas nas metodologias de ensino e aprendizagem de Computação nas principais universidades brasileiras. A aprendizagem ativa é uma opção metodológica que está sendo utilizada com o propósito de garantir que a centralidade do processo esteja mais focada no aluno e em sua aprendizagem, ao invés de se concentrar no professor e na sua capacidade de organizar bem oconteúdo. O problema associado com a aprendizagem ativa é justamente onde está a sua riqueza: existe uma parte significativa do processo de aprendizagem que é controlada e regulada pelo próprio aluno. Em algumas abordagens ativas, essa parte é mais intensa e costuma ser chamada de aprendizagem autodirigida, de forma que o estudante é responsável não apenas por estabelecer seus objetivos de aprendizagem, mas também de buscar pelos recursos apropriados e de avaliar se de fato alcançou ou não satisfatoriamente aos objetivos antes determinados.
%             \end{itemize}

%         \end{itemize}

%         \vspace{0.25cm}
        
%     \end{minipage}
% }

% \vspace{0.3cm}

% \fbox{
%     \begin{minipage}[htb]{0.9\textwidth}
%         \vspace{0.3cm}
                
%         \colorbox{gray!30}{% create a colored box
%             \makebox[0.975\textwidth][l]{% center the text on the page
%                 \ \ \textbf{Further Writing}
%             }
%         }

%         \vspace{0.1cm}
        
%         \begin{itemize}
%             \item Translating these paragraphs to English:
%             \begin{itemize}
%                 \item Por que é importante mencionar essas realidades? Os alunos com nível socioeconômico mais baixo potencialmente percorrem “uma estrada mais difícil” em uma aprendizagem autodirigida. Possivelmente, o poder aquisitivo mais baixo afeta tanto a disponibilidade quanto a diversidade de recursos que eles precisam recorrer para alcançar os objetivos de aprendizagem traçados inicialmente. Recursos, nesse sentido, não compreendem apenas bens materiais como livros, computadores ou automóveis. Competências como o domínio de um outro idioma ou a disponibilidade de tempo integral para se dedicar à graduação são recursos que não estão à disposição para todos os alunos de uma maneira equitativa.
%             \end{itemize}

%         \end{itemize}

%         \vspace{0.25cm}
        
%     \end{minipage}
% }

% \vspace{0.3cm}

% \fbox{
%     \begin{minipage}[htb]{0.9\textwidth}
%         \vspace{0.3cm}
                
%         \colorbox{gray!30}{% create a colored box
%             \makebox[0.975\textwidth][l]{% center the text on the page
%                 \ \ \textbf{Further Writing}
%             }
%         }

%         \vspace{0.1cm}
        
%         \begin{itemize}
%             \item Translating these paragraphs to English:
%             \begin{itemize}
%                 \item Dessa forma, é importante conhecer as relações que existem entre o nível socioeconômico dos graduandos e a forma como a aprendizagem autodirigida efetivamente ocorre. Meu objetivo é investigar essa realidade dentro do Centro de Informática (CIn) da UFPE. O CIn tem um laboratório de pesquisa (o NEXT) exclusivamente focado para a adoção da aprendizagem ativa na graduação de Computação. O NEXT tem experiência de mais de uma década na implantação dessas abordagens nos mais diversos cenários de ensino superior de Computação em Pernambuco.
%             \end{itemize}
%             \item Retaking the MRQ: “How CSE students conduct their SDL in developing countries from the lens of capabilities approach?” and
%             \begin{itemize}
%                 \item (RG1) understanding how CSE students find meaning in their SDL trajectories in developing countries in terms of the capabilities approach, and, after this, 
%                 \item (RG2) identifying the crucial CSE capabilities that emerged from it.
%             \end{itemize}

%         \end{itemize}

%         \vspace{0.25cm}
        
%     \end{minipage}
% }

\subsection{Ontological modality}

As I asserted in Section \ref{reflex-sss:beliefs}, I believe in the existence of a single reality, but there exist multiple interpretations of this one. In this way, it is possible to believe in it and to conciliate with some constructivist assumptions, for instance. An ontological reality, in its strict sense, is “either rejected or at best considered irrelevant” in a constructivist approach \cite[p.~50]{ben-ari:2001}. It is not necessary to reject the existence of a single reality to assume the existence of multiple interpretations in this approach. 

Multiple interpretations do not necessarily lead to multiple realities. But it is impossible to assume a constructivist position without asserting that the people construct the understanding of reality. This is the convergence between my worldview and the constructivist approach.

\subsection{Epistemological modality}

Reality is like a sandbox. Although it is single, it is not static. When people walk into the sandbox, this reality changes. But it remains to be single. Reality is dynamic. Reality is affected by human beings, natural forces, and spiritual beings. But it is single, unique, and shared with all existing things.

But this sandbox is huge and complex. It is like presented in Flatland romance \cite{abbott:1884}. Those creatures access only two-dimensional aspects of reality, while one has three, four, or more dimensions. Like them, we are limited by our human condition. Reality is single but not fully accessed by us.

The Flatland metaphor explains a limited understanding of reality but not explains the multiple interpretations from different people of the same phenomenon. As human beings are not omniscient, they do not have all the information about reality. And due to not having all the information, two people likely do not share the same set of information about the same phenomenon. Two people have different life histories, different family origins, different social conditions, different experiences, and so on. When we put the “Flatland condition” together with the non-omniscience, the existence of multiple interpretations of the same phenomenon is perfectly understandable. It is possible to complicate this situation more if put in perspective that our cognitive structure limits all information we consider to memorize and deal with it appropriately. 

Despite this, it is possible to know something. It is possible to share our interpretations of a little of what we can know. And these shared interpretations enable us to understand this fragment of reality better.

% \vspace{0.3cm}

% \fbox{
%     \begin{minipage}[htb]{0.9\textwidth}
%         \vspace{0.3cm}
                
%         \colorbox{gray!30}{% create a colored box
%             \makebox[0.975\textwidth][l]{% center the text on the page
%                 \ \ \textbf{Further Writing}
%             }
%         }

%         \vspace{0.1cm}
        
%         \begin{itemize}
%             \item Correlating to the seven complex lessons in education for the future \cite{morin:1999}.

%         \end{itemize}

%         \vspace{0.25cm}
        
%     \end{minipage}
% }

% \subsection{Methodological Modality}

% \fbox{
%     \begin{minipage}[htb]{0.9\textwidth}
%         \vspace{0.3cm}
                
%         \colorbox{gray!30}{% create a colored box
%             \makebox[0.975\textwidth][l]{% center the text on the page
%                 \ \ \textbf{Further Writing}
%             }
%         }

%         \vspace{0.1cm}
        
%         \begin{itemize}
%             \item Asking questions about how the research design and methods (i.e., techniques, surveys, interviews, focus groups) place limits on the kinds of data collected (and approaches) and thus exclude other things from being seen or understood. 
%             \item How, for example, might this data have been collected using different techniques and why was one choice made over another?.

%         \end{itemize}

%         \vspace{0.25cm}
        
%     \end{minipage}
% }

% \subsection{Theoretical/Analytic modality}

% \fbox{
%     \begin{minipage}[htb]{0.9\textwidth}
%         \vspace{0.3cm}
                
%         \colorbox{gray!30}{% create a colored box
%             \makebox[0.975\textwidth][l]{% center the text on the page
%                 \ \ \textbf{Further Writing}
%             }
%         }

%         \vspace{0.1cm}
        
%         \begin{itemize}
%             \item Asking questions about how the data are to be analyzed and how by making choices other options are excluded, elided, or never considered. And with different theories and analytic choices we may see things that others might altogether ignore (perhaps the things that make the most difference or the things that truly matter).

%         \end{itemize}

%         \vspace{0.25cm}
        
%     \end{minipage}
% }

% \subsection{Normative modality}

% \fbox{
%     \begin{minipage}[htb]{0.9\textwidth}
%         \vspace{0.3cm}
                
%         \colorbox{gray!30}{% create a colored box
%             \makebox[0.975\textwidth][l]{% center the text on the page
%                 \ \ \textbf{Further Writing}
%             }
%         }

%         \vspace{0.1cm}
        
%         \begin{itemize}
%             \item Asking questions that explore the complex and inevitable dynamic relationship between fact and value \cite[pp.24-29]{sayer:2011}.

%         \end{itemize}

%         \vspace{0.25cm}
        
%     \end{minipage}
% }

% \subsection{Representational modality}

% \fbox{
%     \begin{minipage}[htb]{0.9\textwidth}
%         \vspace{0.3cm}
                
%         \colorbox{gray!30}{% create a colored box
%             \makebox[0.975\textwidth][l]{% center the text on the page
%                 \ \ \textbf{Further Writing}
%             }
%         }

%         \vspace{0.1cm}
        
%         \begin{itemize}
%             \item Facing the choices about how we talk or write about our research participants {\color{red}(Steinmetz, 2004, pp. 380–381)}. 
%             \item Do we use the first or third person, for example? 
%             \item And what is the effect of using the third person when we talk to or about our subjects? \cite[pp.~1-22]{sayer:2011}. 

%         \end{itemize}

%         \vspace{0.25cm}
        
%     \end{minipage}
% }



