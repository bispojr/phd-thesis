\chapter{Reflexivity Essay}
\label{chap:reflex-essay}

An essential step for an investigator in a research project is establishing their position before the existing philosophical perspectives. In a qualitative investigation, the researcher should be aware of your influence on the study object and how (s)he is affected by the research process. \citeonline{probst:2014} catalog a wide range of possible ways the researcher can conduct specific reflexive activities in your research.

Our stance has several differences depending on whether we conduct qualitative or quantitative research. In a quantitative approach, it is more common for the researcher to put away the object to be studied. In this perspective, the tendency is to adopt a neutral position to guarantee as much as possible the absence of biases related to the researcher as a person. The more “experimentalizable” the research is, the more neutral position the researcher should pursue. 

However, when we admit a more interpretive philosophical perspective, for instance, the research nature tells us not to ignore and, therefore, count on our influence before, during, and after our research practice. Thus, one of the ways to reduce bias in a qualitative approach is not to avoid the personal influences in their research but to reveal them explicitly. The hope is to provide a reflexivity essay as an essential and additional data source for readers to consider when appreciating the research report.

In this way, this essay aims to structure a critical self-reflection about my assumptions and worldview as a computing education researcher concerning this thesis. The remainder of this chapter is organized as follows. Section \ref{reflexivity:qualitative-research} presents the reflexivity in qualitative research in a general sense. Section \ref{reflexivity:report} reports my reflexivity from \citeonline[pp.~513-514]{longhofer:2012}'s modalities. %Section \ref{reflexivity:related-work} enumerates and discusses the related works.
At last, Section \ref{reflexivity:final-remarks} summarizes the conclusions and final remarks.
