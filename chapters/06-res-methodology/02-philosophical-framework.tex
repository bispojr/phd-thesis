\section{Philosophical Framework}
\label{res-met:phil-framework}

\fbox{
    \begin{minipage}[htb]{0.9\textwidth}
        \vspace{0.3cm}
                
        \colorbox{gray!30}{% create a colored box
            \makebox[0.975\textwidth][l]{% center the text on the page
                \ \ \textbf{Further Writing}
            }
        }

        \vspace{0.1cm}
        
        \begin{itemize}
            \item Presenting the four major philosophical frameworks possible to follow during a qualitative research.
            \item Describing my position, situating between interpretive and critical frameworks.

        \end{itemize}

        \vspace{0.25cm}
        
    \end{minipage}
}

\begin{table}[ht]
\caption{Epistemological perspectives from their main purposes, types of research, and perceptions of reality.}
\label{tbl:epist-perspectives}
\centering
\rowcolors{1}{}{lightgray}
\begin{tabular}{
    p{1.9cm}|
    m{2.6cm}|
    m{2.7cm}|
    m{2.5cm}|
    m{2.8cm}
}
    \hline
    &
    \multicolumn{4}{c}{
        \textbf{Epistemological Perspectives}
    } \\
    \hline
    &
    \textbf{Positivist/ Postpositivist} & 
    \textbf{Interpretive/ Constructivist} & 
    \textbf{Critical} &
    \textbf{Postmodern/ Poststructural}
    \\
    \hline
    \textbf{Purpose} &
    Predict, control, generalize & 
    Describe, understand, interpret & 
    Change, emancipate, empower & 
    Deconstruct, problematize, question, interrupt
    \\
    \hline
    \textbf{Types} &
    Experimental, survey, quasi- experimental &
    Phenomenology, ethnography, hermeneutic, grounded theory, naturalistic / qualitative &
    Neo-Marxist, feminist, participatory action research (PAR), critical race theory, critical ethnography &
    Postcolonial, poststructural, postmodern, queer theory\\
    \hline
    \textbf{Reality} &
    Objective, external, out there &
    Multiple \mbox{realities} \mbox{(interpretations)}, context-bound &
    Multiple realities (interpretations) situated in political, social, cultural contexts (one reality is privileged) &
    Questions assumption that there is a place where a reality resides: “Is there a there there?”\\
    \hline

    % \multicolumn{1}{c}{
    %     \textbf{\#}
    % } &
    % \multicolumn{1}{c}{
    %     \textbf{Principle}
    % } &
    % \multicolumn{1}{c}{
    %     \textbf{Relation to SDL}
    % } \\
    % \hline     
    % 1 &
    % Problem(s) at the core of the educational proposal. &
    % -\\
    % 2 &
    % Learner as the owner of the problem. &
    % Weakly\\
    % 3 &
    % Authenticity of the problem or task. &
    % -\\
    % 4 &
    % Authenticity of the learning environment. &
    % -\\
    % 5 &
    % Learner drives the problem-solving process. &
    % Strongly\\
    % 6 &
    % Complexity of the problem or task. &
    % -\\
    % 7 &
    % Learners test ideas against alternative views and contexts. &
    % Strongly\\
    % 8 &
    % Reflection on the content and process learned.
    % Strongly \\
    % 9 &
    % Collaboration and multidirectional learning. &
    % Strongly\\
    % 10 &
    % Continuous assessment. &
    % -\\
\end{tabular}

  \par\medskip\ABNTEXfontereduzida\selectfont\textbf{Source:} Adapted from \citeonline[p.~12]{merriam:2016-whatIs}. \par\medskip
\end{table}