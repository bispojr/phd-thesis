\section{Qualitative Research}
\label{res-met:qualitative-research}

I begin this section by telling a true story. My wife was pregnant with my daughter in 2016. Larissa was born on July 23, 2017. She weighed 2.530 kilograms. Her height was 47 centimeters. Although both my wife and I were from Recife, she was born in Jataí, a town of the southwest of Goiás state. According to the gynecologist, our daughter was born in perfect health conditions. The childbirth lasted about two days due to our choice of normal birth. All the information is true and tells us important aspects about the birth of my daughter. But it does not inform us appropriately of the meaning of the birth of my daughter for me.

I used not to desire to be a father. But I used not to reject the idea of being one. When I knew about our pregnancy, I could not process the information instantaneously. I elaborated on this information gradually. I think I was becoming a father as my daughter was growing up in my wife's womb. I realized that I would be a father when I could hear her heartbeat during an ultrasound examination. Hearing those sounds made me realize that I was responsible for generating a new life for this world. And when I got Larissa in my arms soon after her birth, I had a strong feeling of responsibility for caring for such a fragile being. 

This information describes a little of the meaning of Larissa’s birth for me. This is the essential characteristic (and difference) of qualitative research: informing us deeply about the meaning of the phenomenon for a people group. It is possible to investigate Larissa’s birth from the perspective of lethality risk from the measures of her basic indicators, for instance. And surely, the results of this kind of research are fundamental. However, if we need to better understand the meaning of her birth for her parents, for instance, it will be necessary to conduct qualitative research to delineate it in-depth.

It is not so simple to define what qualitative research is. But it is possible to understand it from its characteristics. \citeonline[p.~15]{merriam:2016-whatIs} state that:
\begin{citacao}
    “The following four characteristics are identified by most as key to understanding the nature of qualitative research: the focus is on process, understanding, and meaning; the researcher is the primary instrument of data collection and analysis; the process is inductive; and the product is richly descriptive”.
\end{citacao}
I describe the first two of these characteristics better as follows (Sections \ref{res-met-ss:meaning} and \ref{res-met-ss:primary-instrument}), besides explaining transferability in qualitative research (Section \ref{res-met-ss:transferability}), and situating the philosophical framework that this research rests (Section \ref{res-met-ss:phil-framework}).

\subsection{Focus on Process, Understanding, and Meaning}
\label{res-met-ss:meaning}

This characteristic points to concern about the qualitative researcher's commitment not only to the final data related to a phenomenon. Focusing on the phenomenon process allows us to deepen the observation, detecting more details when describing it. The focus on the people's understanding will enable us to capture the shared understanding or culture of a community. And, lastly, the focus on the meaning allows us to identify the research participants' sense concerning the phenomenon. 

Although \citeonline[p.~7]{denzin:2013} assert that “[...] qualitative researchers study things in their natural settings, attempting to make sense of, or interpret, phenomena in terms of the meanings people bring to them”, depending on the research phase, the researcher’s stance can adjust their looking. There are two possible perspectives for a qualitative researcher: the emic and etic. The emic (or insider) perspective is clearly described by Patton (1985, p. 3, \textit{apud} \citeauthor{merriam:2016-whatIs}, \citeyear{merriam:2016-whatIs}, pp. 14-15) in the following way:
\begin{citacao}
    “[Qualitative research] is an effort to understand situations in their uniqueness as part of a particular context and the interactions there. This understanding is an end in itself, so that it is not attempting to predict what may happen in the future necessarily, but to understand the nature of that setting—what it means for participants to be in that setting, what their lives are like, what’s going on for them, what their meanings are, what the world looks like in that particular setting—and in the analysis to be able to communicate that faithfully to others who are interested in that setting. [...] The analysis strives for depth of understanding”.
\end{citacao}
And the etic (or outside) perspective involves “standing far enough away from or outside of a particular culture to see its separate events, primarily in relation to their similarities and their differences, as compared to events in other cultures” (Pike, 1954, p. 10, \textit{apud} \citeauthor{patton:2015}, \citeyear{patton:2015}, p. 509). 

In this research, I adopted both perspectives depending on the research phase. Until the data collection phase, it prevailed from the emic perspective, including when I reported it in the results section of my thesis. After the data collection phase, it prevailed from the etic perspective, mainly when I made the discussion of the results. The underlying idea was to seek reality as it is, although I believe it is impossible to apprehend it as a whole.

\subsection{Researcher as Primary Instrument}
\label{res-met-ss:primary-instrument}

The researcher is a primary instrument in qualitative research because we are pursuing meaning and understanding. Meaning and understanding are something of human nature intrinsically. Nothing is better than a human to share and capture semantics, considering that artifacts (including computational ones) are excellent syntactic machines only \cite{setzer:2005}.

Bearing that the researcher is a primary instrument to collect data, it is necessary to describe them as well as possible. When we adopt an experimental approach, biases are expected to be tackled by recommending that the researcher distance the research object, aiming not to “contaminate” the research results. However, suppose we admit that the better way to capture meaning and understanding is by employing a human being as an instrument. In that case, knowing more about this instrument will allow us to read the research results they obtained more appropriately. This is one of the reasons for doing positionality and reflexivity essays during qualitative research (see Chapter \ref{chap:reflex-essay}).

% \subsection{Inductive Process}
% \label{res-met-ss:inductive}

% \fbox{
%     \begin{minipage}[htb]{0.9\textwidth}
%         \vspace{0.3cm}
                
%         \colorbox{gray!30}{% create a colored box
%             \makebox[0.975\textwidth][l]{% center the text on the page
%                 \ \ \textbf{Further Writing}
%             }
%         }

%         \vspace{0.1cm}
        
%         \begin{itemize}
%             \item Describing the inductive nature of the process.
%         \end{itemize}

%         \vspace{0.25cm}
        
%     \end{minipage}
% }

% \subsection{Rich Description}
% \label{res-met-ss:rich-description}

% \fbox{
%     \begin{minipage}[htb]{0.9\textwidth}
%         \vspace{0.3cm}
                
%         \colorbox{gray!30}{% create a colored box
%             \makebox[0.975\textwidth][l]{% center the text on the page
%                 \ \ \textbf{Further Writing}
%             }
%         }

%         \vspace{0.1cm}
        
%         \begin{itemize}
%             \item Presenting the power of rich description in qualitative approach.
%         \end{itemize}

%         \vspace{0.25cm}
        
%     \end{minipage}
% }


\subsection{Transferability}
\label{res-met-ss:transferability}

It is crucial to make a distinction between the concepts of reproducibility and transferability. Due to the consensus about some concepts from experimental approaches, qualitative researchers usually adopt the expression "transferability" \cite{finfgeld:2010}. The aim is to situate what conditions are necessary to assume the replication logic under a qualitative lens \cite[p.365]{tuval:2021}. In this direction, "reproducibility" would be more restricted to quantitative approaches and "transferability" to qualitative ones.

\citeonline[p.~677]{kennedy:1979} uses the expression "generalization" but uses modifiers like "nonstatistical" to distinguish it from the experimental paradigm. It is also called for other researchers as analytical generalizations, highlighting the distinction from the statistical one. Both \citeonline{kennedy:1979} and \cite{morse:2002} list a number of ways to make inferences assuming a qualitative stance and showing its extension and limitation. It hopes that good qualitative research should be richly descriptive, aiming to provide conditions for other researchers to transfer the findings from their contexts. 

% \fbox{
%     \begin{minipage}[htb]{0.9\textwidth}
%         \vspace{0.3cm}
                
%         \colorbox{gray!30}{% create a colored box
%             \makebox[0.975\textwidth][l]{% center the text on the page
%                 \ \ \textbf{Further Writing}
%             }
%         }

%         \vspace{0.1cm}
        
%         \begin{itemize}
%             \item Explaining the difference between reproducibility and transferability.
%             \item Presenting the reasons why transferability increases the rigor of qualitative research.
%             \item Taking up the project research goals and why it is more appropriate to conduct it from a qualitative lens.
%             \begin{itemize}
%                 \item Two research goals help to address this question: (RG1) understanding how CSE students find meaning in their SDL trajectories in developing countries in terms of the capabilities approach [thick description], and, after this, (RG2) identifying the crucial CSE capabilities that emerged from it [hypothesis generation].
%             \end{itemize}
%         \end{itemize}

%         \vspace{0.25cm}
        
%     \end{minipage}
% } 

\subsection{Philosophical Framework}
\label{res-met-ss:phil-framework}

This research is situated between two philosophical frameworks: interpretive and critical epistemological perspectives (Table \ref{tbl:epist-perspectives}). Its purpose is (i) to describe and understand how \acrfull{CSE} students conduct their \acrfull{SDL} in developing countries from the \acrfull{CA} lens, (ii) to interpret the results during the discussion phase, and (iii) change the awareness of computing educators as a byproduct of this research. Thus, it is the reason for putting it between these two perspectives (however, the critical traits are more discrete). Its type is qualitative, and I assume multiple interpretations of reality (see Section \ref{reflex-sss:beliefs}).



% \fbox{
%     \begin{minipage}[htb]{0.9\textwidth}
%         \vspace{0.3cm}
                
%         \colorbox{gray!30}{% create a colored box
%             \makebox[0.975\textwidth][l]{% center the text on the page
%                 \ \ \textbf{Further Writing}
%             }
%         }

%         \vspace{0.1cm}
        
%         \begin{itemize}
%             \item Presenting the four major philosophical frameworks possible to follow during a qualitative research.
%             \item Describing my position, situating between interpretive and critical frameworks.

%         \end{itemize}

%         \vspace{0.25cm}
        
%     \end{minipage}
% }