\section{SDL Capabilities}
\label{disc-sec:sdl-capabilities}

Perceiving \gls{SDL} under the lens of competencies is not an innovative approach \cite{patterson:2002,morris:2019,colomer:2021}. In this direction, we can map each stage from Knowles' \gls{SDL} model (see Section \ref{sdl-models-ss:linear}) as a competency to be developed. However, it is necessary to ensure some minimal elements. Bearing in mind that competency can be defined as the intersection of knowledge, skills, and dispositions (e.g., \citeonline{kumar:2023} in \acrfull{CS2023}), it is expected to deepen each of these dimensions for each competency.

Thus, it is not different concerning capabilities. When we decide to transpose each stage from Knowles' \gls{SDL} model as a capability, it is also necessary to develop its minimal three dimensions: (i) achieved functionings (or simply achievements), (ii) means (including goods and services), and (iii) conversion factors (see Figure \ref{fig:grouped-robeyns-representation}).
We already know that the competencies and capabilities approaches have similarities, but there are many distinctions between these two concepts \cite{lozano:2012}, being necessary that we expand and rebase our way to see competency. Thus, I call them \gls{SDL} capabilities set, being composed of (i) taking the initiative, with or without the help of others, in (ii) diagnosing their learning needs, (iii) formulating learning goals, (iv) identifying human and material resources for learning, (v) choosing and implementing appropriate learning strategies, and (vi) evaluating learning outcomes. These six capabilities are interrelated and allow us to analyze Chavo and Quico’s \gls{SDL} under the \acrfull{CA} lens\footnote{It is crucial to highlight that a \gls{CA} analysis tends to be strongly interrelated among the dimensions inside the capability set. Thus, the achievements, means, and conversion factors are expected to appear several times in all capabilities transversally.}. I describe "taking the initiative" \gls{SDL} capability in more detail as follows from its achievements (Section \ref{disc-ss:achievements}), means (Section \ref{disc-ss:means}), and conversion factors (Section \ref{disc-ss:conversion-factors}).

\subsection{Achievements}
\label{disc-ss:achievements}

In an educational equity analysis, it is crucial to identify what functionings, beings and doings (see Section \ref{sen-ss:functioning}), are already achieved by students. As the Universal Monarch of \citeonline[p.~74]{saint-exupery:1943}\footnote{The Universal Monarch is a king from one of the planets visited by the Little Prince, a classic novel written by the French pilot Antoine de Saint-Exupéry.} said, "One must require from each one the duty which each one can perform". All computing educators should consider the current functioning state of their students, seeking to understand what the following steps would be proposed for each one, not only as a learning challenge but also as a fair learning challenge. These achieved functionings are called \glspl{A}, and I identified a list of them (Table \ref{tbl:achievement-list}) concerning "taking the initiative" \gls{SDL} capability from Chavo's and Quico's interviews mainly. 

\begin{table}[ht]
\caption{Achievement list for "taking the initiative" \acrshort{SDL} capability from Chavo's and Quico's data.}
\label{tbl:achievement-list}
\centering
\rowcolors{1}{}{lightgray}
\begin{tabular}{p{0.5cm}p{8.5cm}}
\hline
\textbf{\#} &
\textbf{Achievement}\\
\hline     
A1 &
Realizing the "turning point insight".\\
A2 & 
Possessing specific previous knowledge.\\
A3 &
Having a minimum volition for. \\
A4 &
Being a non-dependent learner. \\
A5 &
Dominating a foreign language. \\
\hline

\end{tabular}
\par\medskip\ABNTEXfontereduzida\selectfont\textbf{Source:} Created by the author (2024). \par\medskip
\end{table}

This is not an exhaustive list (and I am not sure if there is any chance to do it). The aim is to enlighten and expand our vision concerning the reach that an equity analysis can embrace. I discuss two of these achievements, \gls{A}1 and \gls{A}2, in more detail as follows.

\subsubsection{Turning Point (A1)}

\gls{A}1 is "realizing the 'turning point'". Let us see what Chavo answered to a \acrfull{IQ}.6 unfolding question:
\begin{quote}
    \colorbox{black!15}{Me: Ok. [...] So, you are taking a course at the university. So, the professor requi-}
    \colorbox{black!15}{res something of you. So, is there something different that you do because this is} \colorbox{black!15}{a university subject, or does the strategy follow more or less in this same direc-} \mbox{   } \colorbox{black!15}{tion?}
    
    "It depends. It also depends on the scope that he asks us. Because, for example, in the Accounting course, the professor asked us something that he'd never asked people, which was related to building an application using management things. So... for me, it was something that he didn't give us any material for and that, in this course, I had to research it by myself. So, I had to use a different method. \underline{So... from the scope he gave me, I was researching the points}\footnote{See the original excerpt in Brazilian Portuguese in Appendix Section \ref{interview-exc-ss:chavo-iq6}.}" (underlined by me).
\end{quote}

There is a critical momentum, what I am calling turning point insight, that the learner realizes that they need to turn off the receptive (or passive) mode and turn on the active one. This capability to "change the switch" at an appropriate time is directly related to taking the initiative in a \gls{SDL} journey. This feeling helps the learner to regulate their internal dispositions concerning the problem-solving process, putting themselves in a more active role.

Why a computing educator should pay attention to the turning point insight? Because not all computing students have this achievement when they enter a classroom. These students can be required to get the turning point insight when they pursue their \gls{SDL} journey. For instance, probably, Quico does not have this achievement in a well-developed way (see Section \ref{disc-ss:staged-sdl}). Thus, if this assertion is true, the professors who adopt active approaches must map the development level of the turning point insight achievement in their classroom without leaving anyone behind.

Do I, a \gls{CSE} professor, need the turning point insight as a pre-requirement to develop \gls{SDL} activities in my class? If I do, I need to diagnose my class concerning this achievement and propose a learning pathway for all students, considering that the "box distribution" (see Figure \ref{fig:equality-vs-equity}) usually is not well-configured for my \gls{CSE} students.

\subsubsection{Specific Previous Knowledge (A2)}

\gls{A}2 is "possessing specific previous knowledge". Let us see what Quico answered to an \acrshort{IQ}.1 unfolding question:
\begin{quote}
    \colorbox{black!15}{Me: And with this programming side? Did you know anything or not? Or were you} \colorbox{black!15}{a bit of a newbie? How was that?}

    "I had a foundation, so... not too much, but I'd already seen something, do you know? I'd already seen it at other places, so... on YouTube, I'd already seen something related to programming. I focused on one there, which was Python. Afterward, I did a technical program at IF [Federal Institute] of Paulista. It lasted one and a half years. I did Maintenance and Support on Informatics. So it also had... it had programming. So, I got a larger foundation. There was this period that I got to learn by other means, there was this IF period, and now this IS [Information Systems] period, isn't it? So I'd already had there a context, even basic, but I'd already had\footnote{See the original excerpt in Brazilian Portuguese in Appendix Section \ref{interview-exc-ss:quico-iq1}.}".
\end{quote}

Chavo has also computing previous knowledge. Let us what he answered about his expectations concerning the program (\gls{IQ}.2):
\begin{quote}
    "I think I have many expectations with Management [course]. I knew I had, but I didn't know how it was... I'd never seen it a lot. I'd already studied the programming side a little bit before. I'd already lived a bit before. So, I already knew a little about what would happen. But so... this caught my attention a lot because I expected that so... I don't understand what Administration [course] is. I know that I would need to manage something in a project, project lifecycle in some course, but I didn't know how it would be. So I would have to study, so... the beginnings of Administration, Scientific Administration, and this stuff, do you know? So this changed a lot".

    \colorbox{black!15}{Me: Yes. So, didn't you have a clear perception, and this came during the program?}
    
    "Yes"\footnote{See the original excerpt in Brazilian Portuguese in Appendix Section \ref{interview-exc-ss:chavo-iq2}.}.
\end{quote}
In another Chavo's unfolding answer about how he imagines himself after graduating (\gls{IQ}.3), he mentioned his computing previous knowledge:
\begin{quote}
    "I think the technical side was always something that was easier for me because I'd already studied a bit of robotics in high school. It was something that helped me a lot, so the technical side was very good. So the management side is coming more now. Now, I'm getting to improve this\footnote{See the original excerpt in Brazilian Portuguese in Appendix Section \ref{interview-exc-ss:chavo-iq3}.}".
\end{quote}

The specific previous knowledge can help students to taking a differentiated initiative about a problem from a given domain. Thus, the depth of some initiatives depends strongly on "starting knowledge" that a person has. This matter mainly when we refer to the assessment process that may incur at risk to appraise the lack of initiative from the students' achieved results regardless where they depart from.

It is essential to highlight that both Chavo and Quico had their computing previous formation during the basic education. The inclusion of Computing in basic education around the world may contribute to a better performance in \gls{CS} programs and, consequently, reducing the retention and dropout rates. This global phenomenon has a special chapter in Brazil's scenario in the latest years after the consolidation of legislation about the norms to include computing in Brazilian basic education\footnote{Available in \url{https://www.computacional.com.br/docs_oficiais/parecer_homologado.pdf}.}, promoting a better social environment to forge computing capabilities and, as a consequence, a more equitable \gls{CSE}. \citeonline{ribeiro:2023} presents this national standard for school curricula.  

\subsection{Means}
\label{disc-ss:means}

\gls{M} are available resources that a student can have or use to promote their learning. They include not only physical resources (e.g., laptops, other goods) but also services (e.g., public transport, print quota), and even living beings. The \gls{CSE} students' freedom to pursue an expected functioning does not depend only on their achievements, needing to assess what is the availability of means. If two students have the same set of achievements but do not have the same availability of means, it is possible that the same capability is enjoyed by them at different levels. I identified a list of them (Table \ref{tbl:means-list}) concerning "taking the initiative" \gls{SDL} capability from Chavo's and Quico's interviews mainly. I discuss two of these means, \gls{M}1 and \gls{M}2, in more detail as follows.

\begin{table}[ht]
\caption{Means list for "taking the initiative" \acrshort{SDL} capability from Chavo's and Quico's data.}
\label{tbl:means-list}
\centering
\rowcolors{1}{}{lightgray}
\begin{tabular}{p{0.5cm}p{5cm}}
\hline
\textbf{\#} &
\textbf{Means}\\
\hline     
M1 &
Family's \& friends' network.\\
M2 &
Mobility. \\
M3 & 
Digital infrastructure. \\
\hline

\end{tabular}
\par\medskip\ABNTEXfontereduzida\selectfont\textbf{Source:} Created by the author (2024). \par\medskip
\end{table}

\subsubsection{Family’s \& friends’ network (M1)}

\gls{M}1 concerns the friends' network. Chavo began to answer \gls{IQ}.6 in this way: "I try to get in touch with my friends. So... those who I live more together generally"\footnote{See the original excerpt in Brazilian Portuguese in Appendix Section \ref{interview-exc-ss:chavo-iq6}.} (see Section \ref{results-ss:people}). Not having the necessary means can lead you to not pass through the "turning point" (see \gls{A}1, Section \ref{disc-ss:achievements}). As a dam gets in the inevitability of overflowing when the waters exceed its limits, there are critical success factors for \gls{SDL}. Cultivating a friends network allows you to put "more water in this dam", increasing the necessary conditions for passing through the turning point.

\gls{M}1 refers to the family's network, too. During the answer about the choice of his undergraduate program (\gls{IQ}.1), Quico shared:
\begin{quote}
    "\underline{So... we spend time there, my dad and me}, talking and so on. We saw the \acrshort{SiSU} [\acrlong{SiSU}] record, the grades... how was the situation, if it was possible to classify or not. So, by the analysis that we did, Information Systems was a program that was inside what I wanted, that was related to programming, and was also possible for me to classify"\footnote{See the original excerpt in Brazilian Portuguese in Appendix Section \ref{interview-exc-ss:quico-iq1}.} (underlined by me).
\end{quote}
Similarly, the family's network can support students in their initiative taking. Depending on which family context the student is in, they can benefit from deep and affective relationships there, contributing to their self-esteem and self-confidence. Unfortunately, in harmful family contexts, this environment can play the opposite role, not being a source of support and self-fulfillment. It is important to highlight that the promotion of a supportive family context is singular for different  socioeconomic statuses (\acrshort{SES}).

\subsubsection{Mobility (M2)}

\gls{M}2 concerns the mobility of students to arrive at university. Let us see what Chavo said during his answer about his weekly routine (\gls{IQ}.5):
\begin{quote}
    "So when it [my traineeship] ends at 4 pm, I leave, I get dressed literally. It's a bit rushed, but ok. I get dressed. So I have two options. \underline{Or I'm gonna walk} because it's near, so it's possible to walk without problems. I spend 20 to 25 minutes nearly". 

    \colorbox{black!15}{Me: So close.}
    
    "Not very much. \underline{Or I get on the bus}. But the bus goes at 4:20 pm. So I have to hurry up and keep my fingers crossed that it goes at 4:25 or 4:20 pm, but it's possible to get on without problems.
    
    So generally, that’s it. I have classes at college. When I come back, I get on the bus to CCEN [Exact and Natural Sciences Center], and I go to my home. So... I review the things, and I go to bed. Eating and sleeping\footnote{See the original excerpt in Brazilian Portuguese in Appendix Section \ref{interview-exc-ss:chavo-iq5}.}" (underlined by me).
\end{quote}
Concerning this same topic, Quico answered an unfolding question in this way:
\begin{quote}
    \colorbox{black!15}{Me: How do you arrive at home? Do you go by bus?}

    "\underline{I go by car}. I live in Tangamandapio. When the highway is not jammed, it's nearly 22 minutes\footnote{See the original excerpt in Brazilian Portuguese in Appendix Section \ref{interview-exc-ss:quico-iq5}.}" (underlined by me).
\end{quote}

We can unfold three dimensions of students' mobility to reach their educational spaces: distance, means of transport, and financial cost. For instance, the distance between their homes and university can raise issues concerning the duration time of displacement. How much extra class time should I have to conduct my \gls{SDL} appropriately? Suppose a \gls{CS} professor ignores this dimension and assumes that all students have the same availability of extra class time. In that case, there is a chance to misjudge the learning commitment of some students, understanding it as a "lack of a minimum dedication" (which could be an inequality of availability of extra class time for them). Chavo, for example, can spend double of the time than Quico depending on the means of transport used. To what extent should I, \gls{CS} professor, consider aspects like that in my educational equity analysis? It is a big challenge.

Related to means of transport, it can also raise safety issues for students. How safe do I feel to carry personal assets necessary to conduct my \gls{SDL}? Chavo answered an unfolding question of \gls{IQ}.12 in this way:
\begin{quote}
    \colorbox{black!15}{Me: And when you come walking, do you bring your laptop or not? Because it is} \colorbox{black!15}{also a dilemma.}

    "So... so... I don't. It's very difficult for me to come with my laptop. So... I have a friend who lives near here, $\langle$Acapulco$\rangle$. So he passes here sometimes. So when I need my laptop or a ride, he always offers me. So when I'm going with my laptop, something like that, I always go with him or then I get on the bus. But when he gives me a ride, so... it's better, it helps me a lot\footnote{See the original excerpt in Brazilian Portuguese in Appendix Section \ref{interview-exc-ss:chavo-iq12}.}".
\end{quote}
It is not explicit, but it is probable that the trouble faced by Chavo is a safety issue. He lives near the university in a neighborhood known for its high rates of criminality. He discarded the idea of coming walking, limiting himself to using another means of transport for his personal safety, presumably. Indirectly, Chavo does not dispose of all resources that he could have due to safety problems or the impossibility of using another means of transport (e.g., a private car). Surely, these obstacles can hamper students from taking the initiative.

Lastly, the financial cost can raise feasibility issues for students. Is there 'space' in my familiar budget to consider the possibility of using some means of transport, aiming to give me more extra class time to conduct my \gls{SDL}? Let us see what Chavo answered \gls{IQ}.12:
\begin{quote}
    "I got it. I think not... due to study... I can... I can stay with study and no work because it's [a] Federal [university], but it's a bit more complicated. Because... as it's only me and mom at home... and before my job, \underline{it was more difficult because of} \underline{the bus fares} and food at college. And this... because... wanting or not, \underline{bus fares} \underline{cost a lot} and food too. But so... I think that's it. But before... Wait, I think I... Wait a minute. Can you repeat the question, please? Because I got lost.

    \colorbox{black!15}{Me: I can. I'm asking the following: is your job essential to guarantee your studies?}

    Right. Ok... \underline{It would be more related to transportation} and a little to food. Because, so... it's possible to stay [without working]... it's possible. You have to be well-tight. For example, before I began my traineeship, what did I do? I used to do it this way... as going to college is peaceful, with the sun still and so on... all right. \underline{I go walking many times because it's near, so it's already saved one bus fare}. I\\ would only go [from bus] in the coming back. [...] \underline{It's possible to unfurl yourself} \underline{without, but it gets more complicated, do you know?}\footnote{See the original excerpt in Brazilian Portuguese in Appendix Section \ref{interview-exc-ss:chavo-iq12}.}" (underlined by me).
\end{quote}
It seems clear that the mobility financial cost can directly affect students' extra class time. Chavo does not have the same freedom to choose how to enjoy his free time, aiming to take the initiative in his learning compared to other students with better financial terms. 

\subsection{Conversion Factors}
\label{disc-ss:conversion-factors}

\glspl{CF} are conditions that guarantee (or not) a student to convert \glsfirst{M} to an expected functioning (e.g., \gls{CSE} competency). We can analyze them from several levels of perspectives, including personal, environmental, and social ones (see Section \ref{sen-ss:conv-fac}). I identified a list of them (Table \ref{tbl:conv-fac-list}) concerning "taking the initiative" \gls{SDL} capability from Chavo's and Quico's interviews mainly. I discuss one of these conversion factors, \gls{CF}1, in more detail as follows.

\begin{table}[ht]
\caption{Conversion factors' list for "taking the initiative" \acrshort{SDL} capability from Chavo's and Quico's data.}
\label{tbl:conv-fac-list}
\centering
\rowcolors{1}{}{lightgray}
\begin{tabular}{p{0.8cm}p{5.5cm}}
\hline
\textbf{\#} &
\textbf{Conversion Factors}\\
\hline     
CF1 &
Role Model.\\
CF2 & 
First-generation students. \\
CF3 &
Public Safety \\
CF4 &
Pandemic (sanitary crisis).\\
\hline

\end{tabular}
\par\medskip\ABNTEXfontereduzida\selectfont\textbf{Source:} Created by the author (2024). \par\medskip
\end{table}

\acrshort{CF}1 concerns role model. \citeonline[p.~2]{grande:2018} assert that role model is
\begin{citacao}
    "[...] an individual who embodies one or more desirable ways of engaging with the discipline and/or profession. Both the role model and the emulator can be a professional or a student, in any combination".
\end{citacao}
Role model plays a crucial function in computing engagement, signaling a concrete "way of being" to other members of a given community of practice. During the answer about how he imagines himself after graduating (\acrshort{IQ}.3), Chavo shared:
\begin{quote}
    "Today, many things have changed. I intend to deepen in programming: yes... but also in the cloud area, that is something I'm studying more now, and I am interested a lot. I am always interested in the [cyber]security area. So [these] are some areas that I like a lot... and programming. And my expectation is to run after to develop myself: become a junior, middle, and senior [developer], and try to be an expert in the market, a reference. \underline{I think we always want to be a reference in that we like}\footnote{See the original excerpt in Brazilian Portuguese in Appendix Section \ref{interview-exc-ss:chavo-iq3}.}" (underlined by me).
\end{quote}
In his mind, Chavo has a set of role models that help him to build and see himself in the future as a computing practitioner. These role models can serve as a source of motivation for \gls{CSE} students to take the initiative in their learning. In this case, Chavo shared with us a classic career plan route of a developer, expressing part of my understanding about his focus on skill improvement concerning \gls{SDL} goals (see Section \ref{disc-ss:sdl-goals}).

On the other hand, Quico answered \gls{IQ}.3 in this way:
\begin{quote}
    "Hmm... I intend, so... \underline{from the image that I have}, to be working in an area that I like, do you know? So... it's the first thing that comes to my mind... working on something that I like. So, deeper, maybe..., maybe... over time, if I.. in the term, I don't know... I don't know, over the  program... I can begin a traineeship and see the areas inside companies related to programming, to management... that I identify myself more. So I can follow some of them and follow my future. But guessing... clearly, I don't know, so: 'I'm gonna be developing for such company, doing such thing', do you know? \underline{It's unclear...} or I'm gonna be a project manager in a company, do you know?

    But initially, that's it. I'm gonna be working in something that I like, in some company. In the end, I'm already gonna be with a basis and to be a professional, so... well, with a certain experience due to the formation. So... that’s it. It’s unclear in my mind because I don’t have any experience in this area yet, [neither] I’m not looking for a traineeship. \underline{So maybe when I look for a traineeship and begin, it will be} \underline{clearer, do you know?}\footnote{See the original excerpt in Brazilian Portuguese in Appendix Section \ref{interview-exc-ss:quico-iq3}.}" (underlined by me).
\end{quote}
His answer reveals an unclear image of his potential future in computing. I understand that Quico also pursues skill improvement like Chavo, but it seems that he does not appropriate himself concerning the computing "ways of being". Perhaps, Chavo has "more urgency" to imagine himself as a professional than Quico due to \gls{SES} difference between them and, consequently, Chavo would need to begin his career early (and not necessarily as a voluntary and natural professional journey)\footnote{\citeonline{yerdelen:2016}, for instance, introduces the discussion about the career interest of \gls{STEM} low \gls{SES} students in
relation to some equity issues like gender and teaching level.}. 

There are many things to highlight about \gls{CF}1, but I prefer to concentrate on some points here. If we understand that a role model is essential, thus it is necessary to amplify our vision for a diversity of role models. Even inside classic computing communities of practice \cite{wenger:2002}, it is possible to identify a range of options concerning potential computing role models. \citeonline[p.~52]{guzdial:2006} pointed out the importance of creating opportunities to live in these communities from undergraduate studies:
\begin{citacao}
    "[...] Students in computer science classes are rarely working peripherally with real professional software engineers in either design or development, for example. Graduate students, on the other hand, usually are working peripherally with academic researchers, making graduate school more like legitimate peripheral participation.
    
    The best that we in traditional schooling can do is to align our instruction with the students’ perceived community of practice, i.e., the students have to believe that what they are doing and learning will lead them toward central roles in the communities of practice of their choosing. Certainly, there are different communities of practice with which a single course or degree might align, e.g., a student may take a computer science degree in order to become a software engineer, or towards becoming an intellectual property lawyer, or towards some other career where the student believes that deep knowledge of computing is important".
\end{citacao}

Another relevant question concerns the relationship between role models and \gls{CSE} curricula. Are role models a key part of our curriculum? If it is, it must have intentionality to assess the evolution of this capability. This assessment should not get restricted inside a single course, for instance, but passing through the whole \gls{CSE} program. Luckily, some students will have a good set of role models at the end of their undergraduate studies. However, it remains a question: "What is the \gls{CSE} program role (as a necessary part to form this capability in our students)?". A transversal assessment instrument is imperative to follow up their learning journey, providing adequate learning regulation. 

Observing the specific case of \acrfull{UFPE} \acrfull{IS} program, \acrfull{PBL} By-Cycles Framework helps \gls{CSE} students in this direction, promoting this authentic legitimate peripheral participation through a cyclic interaction to several stakeholders as the real client and experts from three different \gls{IS} areas (\acrfull{MIS}, \acrfull{BPM}, and \acrfull{PPM} professors). Authentic formation opportunities can concretely improve the range of computing role models, bringing these computing professionals to closer \gls{CSE} students. One aspect to consider is that this framework, in the way that was implemented in \acrshort{UFPE} \acrshort{IS} program, is more oriented to the skill improvement \gls{SDL} goal, having traits of the critical reflection (e.g., self-assessment activities like \acrshort{PBL}-Test), but no signs of a social emancipation perspective (see Section \ref{sdl-sec:goals}).

Lastly, it does not matter if Quico has \acrfull{SES} higher than Chavo in a democratic education perspective. It is a blunder to suggest that the school community should neglect Quico pursuing "justice" or a "fairer educational environment". The school community (in this case, a university) should always be a partner of a \gls{CSE} student, not their enemy. The struggle for equality of learning opportunities in \gls{CEd} must address and guarantee a common curriculum for all, regardless of their social stratum. If role models are essential to form a computing practitioner, thus every \gls{CSE} student must receive equitable teaching, providing all conditions to have these capabilities in a best-effort approach. \gls{CA} lens direct our looking to guarantee the freedom to \gls{CSE} students to develop a given expected functioning if they want to achieve it.

% I structure this section from Chavo and Quico’s answers for the following question: “Tell me about a typical day of your week”. I triangulate these answers to the findings presented in the previous section, trying to capture potential equity issues. Chavo described his daily routine as follows: 
% \begin{quote}
%     "So, now I work. I wake up at 6 am, about 6h20. So I got until... so, I rest a little, study, have breakfast between 6 and 10 am. 10 am I begin to work until midday. I come back at 1 pm, so from 1 pm to 4 pm, that is the traineeship that I got with the Federal[ Federal is an abbreviation for Federal University of Pernambuco (UFPE), his university.], so it is 6 hours.
    
%     So when it finishes at 4 pm, I get off, I get dressed literally. It's a little busy, but alright. I get dressed, so I have two options: or I go walking because it's closer, so it's easy to walk alright. It takes about 20, 25 little minutes.

%     It's a good time. Or I get a bus. But the bus is at 4h20 pm, so at this moment, it's a hurry to get dressed and cross-fingers to it passes 4h25 or 4h20, but it's easy to get it, no problem. So, in general, that's all. I have a university class. When I get off, I get the bus at CCEN [Center of Exacts and Nature Sciences]. So I go to my home, I review the things and I go to bed... eat and go to bed.

%     The week is about in this way. Weekends, I look for more to see the university things and arrange a bit the things of the week, see if there are any commitments, if I need to go for the company or if I need to go to university to study more, these things, and enjoy myself. That's all\footnote{See the original excerpt in Brazilian Portuguese in Appendix Section \ref{interview-exc-ss:chavo-iq5}.}".
% \end{quote}

% And Quico described his routine in this way:
% \begin{quote}
%     "It's in this way, I think that is... is not so ruled, but it's always the same thing. As I said to you, by the morning I have, from 10 to 14h, I'm free... Not free, I'm at home, but doing university things, maybe some domestic tasks.

%     So my day boils down to that, when it's normal when it has classes. From 10 to 14h, I'm doing any university thing, so when it comes near to 15h, I am getting dressed to come.

%     So I arrive here, I think about 14h, 16h in fact, and I am talking to folks down there. It comes the class time, we go up, so it is from 17h to 20h30 only with classes. So, practically, this boils down to that. An atypical day I think that it would be a no-class day maybe. Because I go to university usually when it happens classes, do you understand?

%     When it doesn't have classes, it doesn't have reason to go, doesn't? But... if I wouldn't have classes and I would come here, it would be more... to solve any project things that need to be everybody in a group together, you know?

%     [At the end of the day,] I arrive about 21h. When I am not talking here until 21h, I arrive at this time, 21h. So... when I'm disposed, I arrive at home... I don't know, I look at class, I look at something of class. So at the end of the day boils down basically to that, to do something of university or then entertainment".
% \end{quote}

% From the findings, there are no critical differences between the most part of \gls{SDL} capabilities of Chavo and Quico. One \gls{SDL} capability shows a significant distinction between them: time as a resource (Section \ref{results-ss:time}). It is important to understand the available time to dedicate to study as a resource, even if it is not perfectly fitted in “identifying human and material resources for learning”. Available time is a transversal resource that perpasses all \gls{SDL} capabilities, and it can affect the equality of learning opportunities.

% It is interesting to highlight that both Chavo and Quico have good grades (Section \ref{results-ss:classroom-data}) at the end of the course. But something important to ask in this scenario is if Chavo did not need to give up other capabilities (e.g., having fun at weekends), aiming to guarantee these SDL capabilities. This capabilities trade-off is an essential point to observe, therefore it can prejudice the student's well-being with severe restrictions on their available time to learn.


% The home-university-home movement is another aspect to highlight. Even Chavo living near to university, he spends in this traject the same time as Quico (who lives in a nearby city). Chavo mentioned during the interview that when he needs to bring your laptop to university, he depends on a friend’s lift to guarantee his personal safety. Probably, the Quico’s private car promotes this personal safety for him. However, the resources should not be the only key to analyzing an equity scenario; some resources can signalize critical conversion factors that can avoid to enjoy an adequate environment for learning.