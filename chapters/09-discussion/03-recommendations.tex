\section{Recommendations}
\label{disc-sec:recommendations}

\gls{RG}3 was addressed by the arrangement of three guidelines and/or recommendations to help educational stakeholders deepen this discussion in their context. I will present each one in the next three following sections.

\subsection{Neutrality}
\label{disc-ss:neutrality}

First, other colleagues and I
structured the discussion about neutrality in \gls{CEd} \cite{bispojr:2022-educomp} from the Brazilian context, bearing in mind that it is not possible to take further steps toward equity awareness without giving up the neutrality presupposition and assuming a minimal set of democratic commitments, intentionalizing their teaching practice. This essay threw light (and some provocations) on the discussion about the supposed political-pedagogic neutrality of professors and its impacts on \gls{CSE}. It presented a little of the Brazilian context concerning the theme of political-pedagogic neutrality and its problematizations. It also exposed some struggles to understand the potential implicit agenda of supposedly neutral discourses and the importance of admitting intentionality in professor practice in \gls{CSE}. This essay still proposed a possible way to build professor identity/ies from a moderate pluralism. We made use of some authors to contribute to the deepening of this discussion, like \citeonline{freire:1996-ped-aut}, \citeonline{skovsmose:2006}, \citeonline{saviani:1994}, \citeonline{hall:1992}, and \citeonline{biesta:2018}.

\subsection{LLM Equity Issues}
\label{disc-ss:llm-equity-issues}

Second, my advisors and I situated emerging equity issues from the use of \acrlong{LLM}s (\acrshort{LLM}) in (computing) education \cite{bispojr:2024-nmp}, emphasizing what we called "Prompt Literacy" and the arising of \gls{LLM}
divide due to the handling of metacognitive competencies. In the second section, we presented the digital divide, listing more common barriers to \gls{ICT} use, the potential mitigation actions for the digital divide problem, and elements to signalize the subjacent structural problem as its roots. In the third section, we described \gls{LLM}, presenting practical examples, as well as showing the opportunities and challenges of its use in educational contexts. In the fourth section, we described the arising of what we call "Prompt Literacy" redeeming the evident evolution (in terms of the complexity and impact of \gls{ICT}) from Web Access Literacy, passing by Search Engine Literacy, 
and arriving in Prompt Literacy. Lastly, we defined \gls{LLM} divide as the gap between those with ready access to \gls{LLM} tools (and the knowledge that they provide access to), and those without such access or skills. We also defined what would be an \gls{LLM} capability under the \gls{CA} lens, listing the primary sources of \gls{LLM} equity issues from this perspective.

\subsection{Equity Analysis Guidelines}
\label{disc-ss:eq-guidelines}

At last, in another work of my advisors and I, we proposed not only a basic discussion about the equity aspects of the adoption of \acrfull{OLEE} \cite{bispojr:2024-online-lab}, but also we listed a set of guiding questions to north an initial equity analysis for collective decision-making in a professor collegiate. Using a storytelling approach, we presented an Engineering Professor called Jirafales\footnote{Teacher Jirafales is one of the characters of Chespirito, a Mexican sitcom written by Roberto Bolaños.} in his journey to adopt \gls{OLEE} in his engineering program. Hypothetical situations (but potentially real) illustrated several equity issues that usually emerges in our teaching practice concerning access, literacy, and social factors. For each of these dimensions, we introduced theoretical constructs about equity from \gls{CA} lens. The idea is to pave the way for an identification with equity agenda, offering the opportunity for a professor to watch themselves as part of Jirafales' dilemmas, feeling his feelings and trying to sketch a practical solution for each fictitious scenario. Empathy and theory walking together: helping each other to forge a new awareness in Engineering Education community. Finally, we created a roadmap comprising of strategic steps (one for each dimension) to follow when a collective educational space needs to conduct an equity analysis. I list the four \glspl{GQ} to consider equity in \gls{OLEE} adoption:
\begin{enumerate}
    \item \underline{Access Dimension}
    \begin{itemize}
        \item[(\gls{GQ}1)] Are there alternatives to learning in case it is impossible to access the OLEE (e.g., a physical lab version)?
        \item[(\gls{GQ}2)] Does my student (or other essential user) have real conditions to access the OLEE outside of university?
    \end{itemize} 
    \item \underline{Literacy Dimension}
    \begin{itemize}
        \item[(\gls{GQ}3)] What are the desired skills and competencies that an undergraduate should have to use my OLEE fully?
    \end{itemize}
    \item \underline{Social Dimension}
    \begin{itemize}
        \item[(\gls{GQ}4)] Is any student group disadvantaged compared to others due to OLEE use?
    \end{itemize}
\end{enumerate}
This book is expected to be published in October 2024 yet\footnote{Springer has already provided the book link: \url{https://link.springer.com/book/9783031707704}.}. 

% "It's not feasible". No, I do not agree with you. Do not ignore this pain and take the next step. Do something.



