\section{SDL Trajectories}
\label{disc-sec:sdl-trajectories}

Aiming to achieve \gls{RG}1, I develop the discussion about Chavo's and Quico's \gls{SDL} trajectories from two perspectives: \gls{SDL} goals (Section \ref{disc-ss:sdl-goals}) and \acrfull{SSDL} stages (Section \ref{disc-ss:staged-sdl}).

\subsection{SDL Goals}
\label{disc-ss:sdl-goals}

Although I presented many possibilities previously concerning \gls{SDL} contexts (e.g., personal, academic, professional), Chavo's and Quico's main focus was from the \underline{skill improvement} perspective (see Section \ref{sdl-sec:goals}). There is a central concern related to their professional improvements, potentially aiming for a better position in the labor market.

It is essential to highlight that the underlying Chavo's and Quico's conception can be that which computing refers not primarily to a personal or social transformation. \acrfull{CEd} is an opportunity to allow them to dispute in a competitive way among the ``players'' in the labor market \cite[p.~428]{bispojr:2024-nmp}. It seems that critical reflection appears as a byproduct of this primary pursuit of a good professional positioning.

It is possible that \underline{critical reflection} is not verbalized due to business culture's tendency to be more ``professional'' during interview moments. I realize that Chavo assumed this standing during most of the interview, signaling a concern to focus on ``professional aspects'' of the answers and, consequently, avoiding a more personal tone that could express some elements in this dimension.

In Quico's case, even when the interview followed a more informal tone, his reported \gls{SDL} case about the development of a Discord bot did not flow to a critical reflection or self-human development \textit{per si}. Quico allowed himself to develop a Discord bot because this effort could promote his skill development.

In both students, the \underline{social emancipation} dimension was not captured, leading me to believe in a perspective more individualistic concerning this \gls{SDL} goals. There are no elements to assert that their group participation or the development of their self-direction sought a struggle or fight against some kind of oppressive situation. Two possibilities to understand this phenomenon better are: (i) assuming that we are living in a post-modern condition and, for this reason, there is an absence of an accepted, cohesive, and coherent society structure, leading to individuals not adhering to solid metanarratives or a "cause" that lead them to want to change or reform the society\footnote{\citeonline[pp.~278-280]{bispojr:2022-educomp} discussed deeper how the perception of identities can affect computing education.}, and (ii) realizing the search for meaning (and values) in life can contribute to the existential vacuum, including in educational contexts \cite{csanli:2021}, leading students to give up having a "solid reason" to live truly.

\subsection{SSDL Stages}
\label{disc-ss:staged-sdl}

The overall impression is Chavo is situated in the \underline{Involved and Self-Directed stages} (Stages 3 and 4) from \gls{SSDL} Grow's axis (Table \ref{tbl:chavo-quico-ssdl-model}, see more in Section \ref{sdl-models-ss:instructional}). There is no strangeness for him concerning requirements to the main steps of \gls{SDL} process. It seems that Chavo handles team activities well and manages these self-study moments reasonably. I felt that Chavo is more independent and appears to assume a considerable level of commitment in his household activities. This behavior seems to facilitate him in a more proactive standing before the demands in a general way. 

\renewcommand\fbox{\fcolorbox{white}{gray!30}}

\begin{table}[ht]
    \caption{Chavo and Quico situated in the \gls{SSDL} Grow’s axis.}
    \label{tbl:chavo-quico-ssdl-model}
    \centering
    \rowcolors{1}{}{lightgray}
    \begin{tabular}{
        c
        c%>{\centering\arraybackslash}p{1.5cm}
        c%>{\centering\arraybackslash}p{3cm}
        c%>{\centering\arraybackslash}p{2cm}
        %p{5.5cm}
    }
    \hline
    \multicolumn{1}{c}{
        \textbf{\#}
    }&
    \multicolumn{1}{c}{
        \textbf{Stage}
    } &
    \multicolumn{1}{c}{
        \textbf{Student}
    } &
    \multicolumn{1}{c}{
        \textbf{Teacher}
    } \\
    \hline
    %\parbox[t]{2mm}{
        \multirow{2}{*}{
            \rotatebox[origin=c]{90}{\fbox{Quico}}
        } &
    %} &     
    Stage 1 &
    Dependent &
    Authority Coach \\%&
    %Coaching with immediate feedback. Drill. Informational lecture. Overcoming deficiencies and resistance.\\
    
    & Stage 2 &
    Interested &
    Motivator, guide \\%&
    %Inspiring lecture plus guided discussion. Goal-setting and learning strategies.\\
    
    \hline
    %\parbox[t]{2mm}{
        \multirow{2}{*}{
            \rotatebox[origin=c]{90}{\fbox{Chavo}}
        } &
    %} &  
    Stage 3 &
    Involved &
    Facilitator \\%&
    %Discussion facilitated by teacher who participates as equal. Seminar. Group projects.\\
    
    & Stage 4 &
    Self-directed &
    Consultant, delegator \\%&
    %Internship, dissertation, individual work or self-directed study-group.\\
    \hline
    
    \end{tabular}
    
      \par\medskip\ABNTEXfontereduzida\selectfont\textbf{Source:} Created by the author (2024). \par\medskip
    \end{table}

In this direction, I want to point out some considerations what I am calling Context-free \gls{SDL}. We need to think about a set of  critical questions about this: (i) Are there cognitive,  metacognitive and motivational capabilities able to transpose to learn new skills across the lifespan? \cite{sheffler:2022}; (ii) Is it possible to think about "meta" \gls{SDL} competencies (or even capabilities) that can serve as a basis to other more contextualized \gls{SDL} journeys (similar to the perspective of upper-level ontologies \cite{niles:2001})?; What is it possible to transpose as a "meta-learning" from a \gls{SDL} journey to another one? These questions touch in a central problem concerning the last assertion about a more independent stance of Chavo can be contextualized in new environments like a \acrfull{MIS} class using a \acrfull{PBL} approach.

Compared to Chavo, I think Quico is situated in the \underline{Dependent and Interested stages} (Stages 1 and 2) from \gls{SSDL} Grow's axis. Quico refers during the interview to many situations in which he needed to validate his "right way" of guiding his \gls{SDL} activities. Quico always validated his choices from third persons (e.g., father, superior) in these cases. I realize this trait is a signal in two directions: (i) firstly, there is more dependency on others, leading to a little developed autonomy (or maybe even a heteronomy); or (ii) secondly, he developed interpersonal intelligence\footnote{Howard Gardner \cite[p.~4]{gardner:1989} structured a theory of multiple intelligences, namely: (i) logical-mathematical, (ii) linguistic, (iii) musical, (iv) spatial, (v) bodily-kinesthetic, (vi) interpersonal, and (vii) intrapersonal.} that leads him to explore more human than non-human resources.

I want to outline some considerations concerning interpersonal intelligence. Depending on how someone usually develops their interpersonal competencies, there are more "suspicions" (or not) related to their self-directedness. If most of someone's interpersonal relationships are inside their family circle, thus this person tends to be considered more dependent and, consequently, less self-directed, being a dependent learner from \gls{SSDL} perspective (see Figure \ref{fig:ssdl-matrix}). This remembers the three theories of life presented by \citeonline[pp.~38,39]{tolstoy:1894}, ranging from individual, passing by tribe (or clan, family, nation), and, finally, coming to a more general principle of life (that encompasses all created things). From a humanistic viewpoint, when a person transcends an individualistic perspective towards broader levels of belonging, they embody self-directedness in its true essence.