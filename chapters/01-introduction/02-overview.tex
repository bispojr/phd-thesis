\section{Overview}
\label{intro-sec:overview}

\gls{CSE} concerns the reflection of equitable variables. Various works in this area approach equity issues like gender \cite{kim:2011}, race/ethnicity \cite{nakajima:2024}, socioeconomic status \cite{parker:2018}, and culture \cite{arawjo:2021}. Equity and diversity also used to be two sides to a story in \acrfull{CEd}, allowing us to see the same problem from these two perspectives \cite{lewis:2019}. However, addressing how to balance different sources of inequities is still an open challenge.

A framework to address this problem is CAPE \cite{fletcher:2021}. This stands for \acrfull{CAPE}. This framework proposes a lens for assessing equity not only in \gls{CSE} but in \gls{CEd} as a whole. Although \gls{CAPE} can map most of the main variables to an equity analysis, the concept of capacity is strongly related to resources, ignoring some essential aspects relative to the real opportunities for a computing student.

Another framework that can address this problem is the \acrfull{CA} proposed originally by Amartya Sen (\citeyear{sen:1992}) and improved by Melanie Walker (\citeyear{walker:2006}) for education purposes. This approach allows us to identify not only the resources that are supposed to be absent in inequity scenarios but also map the capabilities that cannot possibly be developed by a student. Other education fields use the capabilities approach (e.g., Geography \cite{walkington:2018}, but \gls{CEd} still has explored its potentialities marginally.

The \gls{CA} is a theoretical framework based upon two normative claims: (i) the freedom to achieve well-being is of primary moral importance, and (ii) the understanding of well-being is directly related to people’s capabilities and functionings. The freedom of being educated is one of the aims of this perspective, understanding it as a part of the broad problem of liberating people for a fulfilling life.

In this direction, active learning can potentialize the freedom of \gls{CSE} students, promoting more autonomy and crucial soft skills in our complex society. Pedagogical frameworks and methodologies like andragogy \cite{ellis:2002}, problem-based learning \cite{santos:2021}, and peer instruction \cite{bispojr:2021} somehow develop the idea of active learning in this area. These approaches strongly dialog with the constructivism theory (which asserts the students “construct knowledge rather than merely receive and store knowledge transmitted by the teacher” \cite[p.~45]{ben-ari:2001} and, by consequence, with self-directed learning \cite{mccartney:2016}. 

In the \gls{STEM} context, active learning pedagogies have been effective in promoting the increase of learning outcomes \cite{prince:2004}. However, collaborative pedagogies in the \gls{CSE} context has led to marginalization \cite{lewis:2015} like over-dominance concerning student participation. \acrfull{SDL} is a potential equitable practice \cite{anderson:2022}, but there are open challenges to consider regarding when and how to use it \cite{brookfield:1993}. Understanding better how \gls{SDL} effectively occurs in \gls{CSE} students can also contribute to comprehending the potentiality of active learning in terms of capabilities.

In developing countries, other challenges emerge. Beyond the potential inequity sources that emerged from natural diversity in the classroom (e.g., gender, race), structural barriers deepen the situation (e.g., socioeconomic status, poverty). In African countries, for instance, \gls{CAPE} framework is used to analyze equity issues in \gls{CEd} \cite{tshukudu:2023}. Although the authors highlight the strengths of its use, they also point out some limitations: 
\begin{citacao}
    ``The \gls{CAPE} framework helps map the progression from 'Capacity for' to 'Experience of' computer science education as a route to equity, but in order to support development in low and middle income countries, it may be helpful to have the capacity level finely grained'' \cite[p.~1]{tshukudu:2023}.
\end{citacao}
Maybe the capability approach can help to fill some gaps during equity analysis using only the \gls{CAPE} framework.

 In this way, the proposed research helps to establish a process to identify the crucial \gls{CSE} capabilities in the context of self-directed learning in developing countries. The fundamental presupposition is to ensure fair and equitable \gls{CSE}, mainly in the Global South. However, how do we propose the actions and policies needed to mitigate and, if possible, eliminate the sources of unfairness from an interrelated and multifactorial perspective of equity issues (e.g., race, gender, socioeconomic status)? One way is to assess the educational scenario from the capabilities approach.