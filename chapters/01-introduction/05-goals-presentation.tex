\section{Goals and Presentation}
\label{intro-sec:goals-pres}

The \gls{MRQ} of this \gls{Ph.D.} thesis is 
\begin{itemize}
    \item[\textbf{(\gls{MRQ})}] ``How do \gls{CSE} students conduct their \gls{SDL} in developing countries from the \gls{CA} lens?''.
\end{itemize}
Three research goals (\glspl{RG}) help to address this question: 
\begin{itemize}
    \item[\textbf{(\gls{RG}1)}] understanding how \gls{CSE} students build their \gls{SDL} trajectories in developing countries; 
    \item[\textbf{(\gls{RG}2)}] mapping the main elements of \gls{SDL} capabilities observed in \gls{CSE} students in developing countries; and
    \item[\textbf{(\gls{RG}3)}] recommending guidelines to (\gls{CSE}) educational stakeholders concerning how to consider effectively equity issues and active learning from the \gls{CA} lens.
\end{itemize} 

From this point, the thesis will be written in the first person where applicable. My philosophical basis of research allows me to seek methodological rigor without writing it in the third person, which is usually necessary when looking for more objectivity. It is possible to guarantee objectivity without forcing supposed neutrality (\citeauthor{saviani:1994}, \citeyear{saviani:1994}, p.~76; \citeauthor{bispojr:2022-educomp}, \citeyear{bispojr:2022-educomp}). I will develop this idea better in Chapter \ref{chap:reflex-essay}.

The remainder of this thesis is divided as follows. Chapters \ref{chap:sdl} e \ref{chap:equity} present the two central concepts of this thesis: self-directed learning and equity, respectively. Chapter \ref{chap:rel-work} identifies and discusses the related work through a mapping review. Chapter \ref{chap:reflex-essay} positions a reflexivity essay, providing my assumptions and worldview during my \gls{Ph.D.} journey. Chapter \ref{chap:res-methodology} establishes the rationale for the methods, discussing their appropriateness in this research. Chapter \ref{chap:res-design} structures the research design, pointing out the specificities concerning the method application. Chapter \ref{chap:results} presents the results obtained from the data collection. Chapter \ref{chap:discussion} discusses the results, searching for answers to the research questions. Chapter \ref{chap:final-remarks} summarizes the final remarks, shimmering the main findings and potential future works. Finally, Appendix \ref{chap:appendix-a} provides the trajectory of my latest published research and other essential artifacts used during this research (Appendixes \ref{chap:socio-quest}, \ref{chap:appendix-scripts}, \ref{chap:data-charting}, \ref{chap:interview-excerpts}, and \ref{chap:pbl-see-charts}).

