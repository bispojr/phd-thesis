\section{Motivation Problem}
\label{intro-sec:mot-prob}

Imagine three children and one problem. You must decide who (Anne, Bill, or Carla) should get a mini keyboard built essentially from Arduino components. Anne claims the mini keyboard because she is the only one capable of playing it (and the others do not deny this fact). In her vision, denying the mini keyboard to the only person who really knows how to play it would be unfair. If this were all you knew, you would have a solid reason to give Anne the mini keyboard. 

Now, imagine a second scenario. Bill claims the mini keyboard because he is the only one so poor that he has no toys. The mini keyboard allows him to play (and the others admit they are more prosperous than him and dispose of a good variety of toys). If you listened only to Bill, you have a solid reason to give the mini keyboard to him. 

Imagine, at last, the third scenario. Carla says that she built the mini keyboard with her own hands. She worked hard for many months for this. And only when Carla finished making it, the other children claimed the mini keyboard. If you listened only to Carla, it would be plausible to agree that she should be able to use something she made.

These analogies are an adaptation of Amartya Sen’s example (\citeyear{sen:2009}), identifying the difficulties of choosing the fairer option. Depending on your philosophical basis (e.g., utilitarianism, libertarianism, economic egalitarianism), the decision may differ in each case but is still ``obvious'' from each viewpoint. We can transpose this problem to \gls{CSE}. How could \gls{CSE} stakeholders appropriately consider the various equity issues that emerge from a diverse class? How could they balance race, gender, and socioeconomic issues, for instance? How can this be done in the active learning context, where student engagement is highly expected?