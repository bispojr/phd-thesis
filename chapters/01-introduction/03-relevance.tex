\section{Relevance to Computing Education Practitioners}
\label{intro-sec:rel-computing}

It is possible to highlight three directions concerning the relevance of this \gls{Ph.D.} thesis to \gls{CEd} practitioners. First, this thesis contributes to forging awareness about equity issues in \gls{CEd} \cite{bispojr:2022-educomp}. \gls{CEd} practitioners usually do not have in their previous formation an adequate space to discuss and deepen the discussion about equity issues in computing contexts. Writing materials and papers contextualizing into \gls{CEd} allows them to visualize the applicability in their professional places better.

Second, this thesis contributes to the provisioning of constructs to analyze equity in a computing class \cite{bispojr:2024-nmp}. It is essential to increase awareness to promote the intrinsic will towards an educational change. But this awareness can be fruitless if \gls{CEd} practitioners cannot verbalize it using a set of equity constructs. \gls{CA} provides this set, and, during this thesis, it is possible to realize how to identify each one of them inside a situated \gls{CEd} context.

Third, and last, this thesis refers to a pragmatic instantiation of equity discussions in \gls{CSE} (mainly in \cite{bispojr:2024-online-lab}). The proposition of a set of guiding questions to orientate an initial equity analysis for an Engineering collective decision-making of professors serves this purpose, provoking them not only to change their standing but also change their actions through the following of this propositional pathway. These recommendations can easily be transposed from engineering education to \gls{CEd}.
                