\section{Some SDL Relations}
\label{sdl-sec:relations}

Researching \gls{SDL} is strategic due to several relations with other educational concepts and perspectives. This section presents relations from \gls{SDL} to learning in adulthood (Section \ref{sdl-relations-ss:andragogy}), problem-based learning (Section \ref{sdl-relations-ss:pbl}), and self-regulated learning (Section \ref{sdl-relations-ss:srl}).

\subsection{SDL and Learning in Adulthood}
\label{sdl-relations-ss:andragogy}

In the major part of the 20th century,  learning in adulthood was considered a synonym for adult education. Bearing in mind that the understanding of what adulthood is and what an adult can do and be has changed over time (e.g., the conception about the emerging adulthood \cite{parameswaran:2020}), learning in adulthood is an organic area that has been evolving continuously. Currently, learning in adulthood encompasses many learning situations (e.g., professional, personal learning needs) and learning spaces (e.g., formal, nonformal, informal learning), revealing that adult education is one of the various possibilities to address the phenomenon. Aiming to illustrate the rich relations between \gls{SDL} and learning in adulthood, I will focus on Knowles' andragogy and Freire's pedagogy of the oppressed.

Knowles' andragogy \cite{knowles:2005} is based on the need to understand how adults learn differently than children. Andragogy usually rests on six assumptions involving epistemological and anthropological perspectives about the adult and their learning. One of these assumptions is that an adult would tend to a self-direction stance instead of a more dependent one \cite[p.~84]{merriam:2007}. This self-direction propensity would conduct adult to  \gls{SDL} naturally. Thus, teaching adults considering their current human development stage and phase of social life requires adopting pedagogical perspectives toward \gls{SDL}.

Freire's pedagogy of the oppressed \cite{freire:2000-oppressed} arose from a different context of andragogy, rooted in poverty, illiteracy, and oppression among adults. Freire does not differentiate personal empowerment and social transformation, enabling the adult learner "to speak a true work" and "transform the world" simultaneously \cite[p.~87]{freire:2000-oppressed}. This pedagogy becomes concrete through a robust dialogic process, giving up from authoritarian perspectives (called by him "banking education") to a collective and situated construction of both curricula and learning. Thus, inside Freire's pedagogy, \gls{SDL} always appears as a means to promoting critical reflection, putting all stages of Knowles' \gls{SDL} in a sociocultural pedagogical action, reframing the classical notion of an individual adult learner. \citeonline{owen:2002} puts Freire's \gls{SDL} in adulthood learning beside Jack Merirow and Stephen Brookfield as critical perspectives.

\subsection{SDL in Problem-Based Learning}
\label{sdl-relations-ss:pbl}

\gls{SDL} plays a crucial role in the \gls{PBL} approach. \citeauthoronline{savery:1995} (\citeyear{savery:1995}, p.~4) define \gls{PBL} as "a general model [that uses] authentic problems as the stimulus for and organizer of learning activities and the learners work in small collaborative groups". \gls{PBL} has been adopted in Computing Education for a long time \cite{santos:2021}, and there are multiple paths to incorporate it from a single course scope to an entire program curriculum.

Aiming to assess the \gls{PBL} approach, \citeauthoronline{savery:1995} assert that it should have "both a critique of performance and suggestions of ways to improve in three areas: self directed learning; problem solving; skills as a group member". Thus, it is possible to realize that, for them, \gls{SDL} is one of the three essential areas when we adopt \gls{PBL}.

Another way to realize the importance of \gls{SDL} in \gls{PBL} is by analyzing the \gls{PBL} principles. Santos and colleagues \cite{santos:2013,santos:2014,santos:2016,arruda:2019} extracted ten principles of the \gls{PBL} approach from the main \gls{PBL} literature, including the seminal papers of \citeauthoronline{savery:1995}. These ten \gls{PBL} principles are listed in Table \ref{tbl:pbl-principles}. Of ten principles, five are strongly related to \gls{SDL} (Principles 2, 5, 7, 8, and 9). Hence, this evidences how \gls{SDL} is fundamental in the \gls{PBL} conception.

\begin{table}[ht]
\caption{Ten \acrshort{PBL} principles and their relation to \gls{SDL}.}
\label{tbl:pbl-principles}
\centering
\rowcolors{1}{}{lightgray}
\begin{tabular}{
    >{\centering\arraybackslash}p{1cm}
    p{9.7cm}
    >{\centering\arraybackslash}p{3.5cm}
}
\hline
\multicolumn{1}{c}{
    \textbf{\#}
} &
\multicolumn{1}{c}{
    \textbf{Principle}
} &
\multicolumn{1}{c}{
    \textbf{Relation to SDL}
} \\
\hline     
1 &
Problem(s) at the core of the educational proposal. &
-\\
2 &
Learner as the owner of the problem. &
Strong\\
3 &
Authenticity of the problem or task. &
-\\
4 &
Authenticity of the learning environment. &
-\\
5 &
Learner drives the problem-solving process. &
Strong\\
6 &
Complexity of the problem or task. &
-\\
7 &
Learners test ideas against alternative views and contexts. &
Strong\\
8 &
Reflection on the content and process learned.&
Strong \\
9 &
Collaboration and multidirectional learning. &
Strong\\
10 &
Continuous assessment. &
-\\
\hline

\end{tabular}

  \par\medskip\ABNTEXfontereduzida\selectfont\textbf{Source:} \citeonline{santos:2014}. \par\medskip
\end{table}

Other relations can be identified between \gls{SDL} and \gls{PBL}. \citeonline[p.~193]{leary:2019} presented evidence supporting some claims that
\begin{citacao}
    "[...] PBL promotes self-directed learning both as a process within PBL and as an outcome of effective PBL interventions. Further, self-directed learning is mediated heavily by student and teacher perceptions, by environmental factors, and by underlying models that are used (or not) as part of a larger intervention".
\end{citacao}
The authors also mention the benefit of \gls{SDL} development in the \gls{PBL} process:
\begin{citacao}
    "The development of effective self-directed learning skills includes self-assess\-ment and flexible knowledge so that the student understands their personal learning needs and where to find and use appropriate information for problem-solving. [...] They need to set goals and be able to identify their knowledge gaps, strategize how to reach their goals, implement the plan, and assess if they have reached their goal. The collaborative part of self-directed learning encompasses all aspects of working in a group and, as students commence work in PBL environments, working through problems they build their self-directed learning skills" \cite[p.~188]{leary:2019}.
\end{citacao}
Lastly, \citeonline[p.~183]{leary:2019} show the difference between \gls{SDL} and self-regulated learning but attest to the similarity between them in \gls{PBL} contexts:
\begin{citacao}
    "Self-regulated learning focuses on narrow micro-level constructs with tasks typically set by a teacher in a formal learning environment, while self-directed learning stems from adult education and involves broader macro-level constructs initiated by students. Zimmerman and Lebeau contend that in context of PBL, that self-regulated learning and self-directed learning are highly similar and oftentimes the literature uses the terms interchangeably".
\end{citacao}


\subsection{SDL and Self-Regulated Learning}
\label{sdl-relations-ss:srl}

\gls{SRL} is "an active, constructive process whereby learners set goals for their learning and then attempt to monitor, regulate, and control their cognition, motivation, and behavior, guided and constrained by their goals and the contextual features in the environment" \cite[p.~453]{pintrich:2000}. As mentioned in the previous section, \gls{SRL} has an interchangeable use to \gls{SDL} in \gls{PBL} context, but, although its definition closely relates to \gls{SDL} one, there are differences between them.

\citeonline[p.~192]{saks:2014} present a list of differences between these two concepts ranging from (i) origins, (ii) kind of educational space, and (iii) construct level. \gls{SDL} came from Adult Education, while \gls{SRL} came from Educational and Cognitive Psychology. \gls{SDL} is most used to describe activities outside the traditional school environment, while \gls{SRL} is more studied in formal spaces. \gls{SDL} involves a broader macro-level construct initiated by students, while \gls{SRL} focuses on narrow micro-level constructs.

The relation between them is better described by \cite[p.~192]{saks:2014}:
\begin{citacao}
    "Self-directed learning has been considered a broader construct encompassing self-regulated learning as narrower and more specific one. SDL has also been treated as a broader concept in the sense of learner's freedom to manage his learning activities and the degree of control the learner has. In SDL this is the learner who defines the learning task, in SRL it may also be a teacher. [...] Self-directed learning may include self-regulated learning but not the opposite. In other words, a self-directed learner is supposed to self-regulate, but a self-regulated learner may not self-direct. From this point of view, self-directed learning deals more with subsequent steps in the learning process. Providing students with opportunities for self-directed practice can help to improve their self-regulation".
\end{citacao}
\citeonline[p.~418]{loyens:2008} go to same direction concerning the \gls{SDL} coverage:
\begin{citacao}
    "In sum, the concept of SDL is broader than SRL. SDL as a design feature of the learning environment stresses students' freedom in the pursuit of their learning. 
    
    [...] As such, a closer examination of both SRL and SDL as learning processes brings the issue of student control over the learning task to the fore. Clearly, both SDL and SRL carry an element of student control. However, the degree of control the learner has, specifically at the beginning of the learning process when the learning task is defined, differs in SDL and SRL. In SDL, the learning task is always defined by the learner. A self-directed learner should be able to define what needs to be learned. [...] In this sense, SDL can encompass SRL, but the opposite does not hold. SRL seems more concerned with the subsequent steps in the learning process such as learning goals and strategies, while SDL clearly provides a crucial role for the learner at the outset of the learning task".
\end{citacao}