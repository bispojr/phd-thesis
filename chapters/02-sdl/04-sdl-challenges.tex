\section{Some Research Challenges and Opportunities}
\label{sdl-sec:challenges}

\citeonline[p.~128]{merriam:2007} listed some challenges in \gls{SDL} research, posing these questions bearing in mind adult education mainly. I highlight three of them:
\begin{itemize}
    \item[(i)] "Are there public policy issues at the national, state, or local level related to \gls{SDL}? If so, what roles could adult educators play in advocating and developing such policies?
    \item[(ii)] To what extent is \gls{SDL} situational or cultural?
    \item[(iii)] How do cultural and contextual factors shape \gls{SDL}?”
\end{itemize}
This \gls{Ph.D.} thesis, when addressing equity issues in a developing country context, aims to explore better these three questions in a \gls{CSE} scenario.

Several studies addressed the meeting between \gls{SDL} and \gls{CEd}, signaling good promissory ways of researching this topic. \citeonline{mccartney:2016} investigated the reasons computing students learn on their own, interviewing seventeen students from different backgrounds. \citeonline{tagare:2023} reviewed the literature looking for the dispositions that computing professionals value, identifying two of five themes strongly related to \gls{SDL} (self-regulation and lifelong orientation). \citeonline{cawley:2014} incorporated what they called \gls{SDL} pedagogy in computing classroom in \gls{PBL} software engineering courses. Lastly, \citeonline{maphalala:2024} faced \gls{CSE} under the \gls{SDL} lens, reviewing existing research in this area.