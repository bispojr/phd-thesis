\chapter{Conclusions}
\label{chap:final-remarks}

 This work investigated how \acrfull{CSE} students conduct their \acrfull{SDL} in developing countries from the \acrfull{CA} lens (\acrfull{MRQ}). Three \acrlong{RG}s (\acrshort{RG}) helped to address this question in a qualitative approach: (i) understanding how \gls{CSE} students build their \gls{SDL} trajectories in developing countries (\gls{RG}1), (ii) mapping the main elements of \gls{SDL} capabilities observed in \gls{CSE} students in developing countries (\gls{RG}2), and (iii) recommending guidelines to (\gls{CSE}) educational stakeholders concerning how to consider effectively equity issues and active learning from the \gls{CA} lens (\gls{RG}3).

To achieve \gls{RG}1, I structured the perceptions of two \gls{CSE} Brazilian undergraduates about their \gls{SDL} trajectories, being each one from the lowest and highest \gls{SES} of their classroom, respectively. I collected and analyzed interviews primarily to construct the understanding of perceptions (Section \ref{res-sec:interviews}) with the help of other data sources to better situate the findings. At last, I discussed the results from the perspectives of \gls{SDL} goals (Section \ref{disc-ss:sdl-goals}) and \acrfull{SSDL} stages (Section \ref{disc-ss:staged-sdl}), locating their \gls{SDL} trajectories through this looking. 

To achieve \gls{RG}2, I analyzed the results obtained in \gls{RG}1, but from the \gls{CA} perspective (Section \ref{disc-sec:sdl-capabilities}). I proposed a new concept of \gls{SDL} capabilities, (i) identifying six of them from Knowles' process (Figure \ref{fig:sdl-process}), and (ii) examining in detail each one from the achievements, means, and conversion factors dimensions.

 Finally, to achieve \gls{RG}3, I arranged four guidelines and/or recommendations to help educational stakeholders deepen this discussion in their context. First, other colleagues and I structured the discussion about neutrality in \acrfull{CEd} \cite{bispojr:2022-educomp} from the Brazilian context, bearing in mind that it is not possible to take further steps toward equity awareness without giving up the neutrality presupposition and assuming a minimal set of democratic commitments, intentionalizing their teaching practice. Second, my advisors and I situated emerged equity issues from the use of \acrlong{LLM}s (\acrshort{LLM}) in (computing) education \cite{bispojr:2024-nmp}, emphasizing what we called "Prompt Literacy" and the arising of \gls{LLM} divide due to the handling of metacognitive competencies. At last, in another work of my advisors and I, we proposed not only a basic discussion about the equity aspects of the adoption of online laboratories in Engineering Education \cite{bispojr:2024-online-lab}, but also we listed a set of guiding questions to north an initial equity analysis for collective decision-making in a professor collegiate. %At last, I created a roadmap comprising of steps to follow when a collective educational space needs to conduct an equity analysis.

 In the next sections, I present the contributions and implications of my \gls{Ph.D.} journey (Section \ref{conclusions-sec:contrib-impl}), its limitations (Section \ref{conclusions-sec:limitations}), future directions and challenges (Section \ref{conclusions-sec:challenges}), and, lastly, my final remarks (Section \ref{conclusions-sec:my-final-remarks}).

 \section{Contributions and Implications}
 \label{conclusions-sec:contrib-impl}

The contributions can be divided into two groups: (i) those that originated directly from \gls{MRQ} and all \glspl{RG}, and (ii) the remaining ones that originated during the \gls{Ph.D.} period. This division is necessary because the adopted thesis format is a classic monograph, and not a paper-based thesis \cite{kubota:2021}, for example. Thus, these two kinds of contributions need to be encompassed to highlight the research contributions and also fruits fostered by \acrfull{CIn} postgraduate program or even resulting from other \gls{CEd} challenges that emerged during the \gls{Ph.D.} period.

\textbf{First} contribution (emerged from \gls{MRQ}) in the group (i) refers to the use of \gls{CA} as an equity theoretical framework in computing research, and \gls{CEd} mainly. There are works approaching \gls{CA} and technological areas in a generic way (e.g., Engineering \cite{fernandez:2014,odonovan:2020}), but not focusing on computing. The thesis as a whole, in a monograph format, contributes to introducing this theoretical novelty. \textbf{Second} contribution in the group (i) (emerged from \gls{RG}2) refers to the proposition of a new concept called \gls{SDL} capabilities, providing a lens to assess equity in active learning scenarios \cite{bispojr:2024-isdls}. This is a contribution to the Education area in general. \textbf{Third}, and last, contribution in the group (i) (emerged from \gls{RG}3) refers to a pragmatic instantiation of equity discussions in \gls{CSE} (mainly in \cite{bispojr:2024-online-lab} but also in \cite{bispojr:2022-educomp,bispojr:2024-nmp}). The proposition of a set of guiding
questions to orientate an initial equity analysis for an Engineering collective decision-making of professors serves this purpose, provoking them not only to change their standing but also change their actions through the following of this propositional pathway.

\textbf{First} contribution in group (ii) refers to all other relevant \gls{CEd} publications during the \gls{Ph.D.} period \cite{bispojr:2024-nmp,bispojr:2024-urca,feitosa:2024,cavalcanti:2024,pereira:2024,melo:2024-horizontes,boaventura:2024-sbgames,
boaventura:2023,esmeraldo:2023,freire:2023-rsc,freire:2023-encompif, santos:2022,bispojr:2022-educomp,esmeraldo:2022,bispojr:2021,bispojr:2021-educomp,bispojr:2021-wei,bispojr:2020-tec}. Lastly, \textbf{second} contribution in group (ii) refers to other relevant computing publications in the same period \cite{cavalcanti:2024-ieee,bispojr:2023-edi,bispojr:2023-rbie,sansil:2023,lima:2022,bispojr:2022-snee}.

I can list \textbf{two implications} of this research. \textbf{First}, it can support the Brazilian discussion in computing (and \gls{CEd}) spaces concerning \gls{DEI} agenda at more diverse levels. For example, at a local level, \gls{CoDi} is a \gls{CIn} group that discusses and fosters this agenda in a computing college at Pernambuco State. At a national level, in \gls{CEd} area, \gls{IDEA} working group \cite{pereira:2024,melo:2024-horizontes} develops this agenda inside the \gls{CEduComp} from \acrfull{SBC}. Still, at a national level, \gls{SBC} created the \gls{CIDE} to put this agenda inside the society at its higher organization level. \textbf{Second}, it can provide a solid material to \gls{CEd} stakeholders proposing and idealizing equity-minded syllabus \cite{anderson:2023,gama:2024} and/or curricula \cite{karimi:2024} in computing programs. This discussion is crucial not only at the Brazilian higher education level \cite{moro:2022} but also at the basic one \cite{falcao:2021}, bearing in mind that it is 
in operation, currently, a new National Standard for School Curricula \cite{ribeiro:2023}.

 \section{Limitations}
 \label{conclusions-sec:limitations}

One limitation is the epistemological nature of this research. Due to the choice of qualitative research as a methodological presupposition, it is not possible to make statistical generalizations from my findings. However, we can make analytic generalizations \cite{kennedy:1979}, signaling for future contextualizations.

Another limitation is underusing all collected data in this research, impeding the exploration and enriching of the results to understand the phenomenon better. For instance, I interviewed 12 students altogether. Two of them were only chosen as purposeful samples, but I could deepen my knowledge of these two by triangulating the sample findings with other students' data.

The last limitation is the participation withdrawal during the data collection. The first signaling came from 30 students (signing the consent form). Unfortunately, only 15 answered the socioeconomic questionnaire, and 11 participated in interviews. Analyzing from the gender lens, for example, three women answered the socioeconomic questionnaire, and none participated in interviews. I get a feeling that it emerged hesitation among them (being woman or not), maybe with a fear that their course performance could be affected depending on what they would say to me during the research.

 \section{Future Directions and Challenges}
 \label{conclusions-sec:challenges}
 
\subsection{Future Directions}

One of the future directions of this research is to investigate this phenomenon from the perspective of other stakeholders. Equity is complex and requires a multidimensional evaluation to guarantee more qualified information to provide a better decision. This research focuses on the students' watchful eyes, but it can be enriched with other fertile data sources like professors, educational managers, and other stakeholders.

Another future work resides in the examination of collected data not yet explored properly during the \gls{Ph.D.} period. As mentioned previously (Section \ref{conclusions-sec:limitations}), it is possible to delve into other interview transcripts or even collected documents to deepen the knowledge about this phenomenon by triangulating the sample findings with other student data. 

In Section \ref{intro-sec:overview}, I mentioned that \gls{CA} can help to fill some gaps during equity analysis using only the
\gls{CAPE} framework. In this direction, I shimmer two exciting research directions. First, comparing \gls{CA} and \gls{CAPE} equity analysis in detail, pointing out the strengths and weaknesses of each one. Second, it is promising to merge \gls{CAPE} with \gls{CA}, aiming to usufruct the strengths of the two approaches, creating a hybrid framework.

Lastly, the scoping mapping review in this \gls{Ph.D.} research (Chapter \ref{chap:rel-work}) stopped in the second iteration. It would be crucial to \gls{CEd} community to continue the snowballing approach, conducting the remaining iterations. The \acrfull{SRQ} and \acrlong{DSRQ}s (\acrshort{DSRQ}) coverage contribute significantly to mapping the broader area, paving the way for future works.


% \vspace{0.3cm}

%  \fbox{
%     \begin{minipage}[htb]{0.9\textwidth}
%         \vspace{0.3cm}
                
%         \colorbox{gray!30}{% create a colored box
%             \makebox[0.975\textwidth][l]{% center the text on the page
%                 \ \ \textbf{Further Writing}
%             }
%         }

%         \vspace{0.1cm}
        
%         \begin{itemize}
%             \item Finish mapping review.
%         \end{itemize}

%         \vspace{0.25cm}
        
%     \end{minipage}
% }

\subsection{Challenges}

 One of the challenges is continuing this research toward what is called "negative capabilities" \cite{unterhalter:2020}. The idea of negative capability refers to situating some limits of what is measurable and framing aspects of the education process associated with uncertainty and public scrutiny of complexity. For example, how can we investigate when a student "gives up" from a capability aiming to guarantee another one? There are scenarios in South Global contexts where students usually sacrifice their well-nourishment, aiming to get money or time to accomplish a certain kind of academic demand. Knowing better in which settings these "trade-offs" occur in \gls{CEd} matters.

 Another promissory way to investigate in depth is equity in \gls{CEd} under the lens of existential analysis. Equity issues are usually related to existential adversities, leading all involved people to reflect (and even question) the \gls{MiL} \cite{manco:2021}. In some scenarios, students intentionally sacrifice some of their capabilities on behalf of promoting \gls{MiL} like values, a sense of purpose, or even a reason to continue to live. Beyond this, it is possible to use Durkheim's Fatalistic Suicide Typology \cite{godor:2017} better to understand \gls{CEd} dropout phenomenon, extending the \gls{MiL} idea to a more specific "meaning in academic life".

 \section{My Final Remarks}
 \label{conclusions-sec:my-final-remarks}

 These are my final remarks. I tried to be a researcher in a holistic way within my limitations but with my best effort. I searched to be active in both serving my research community and sharing part of my research results in several scientific venues. I tried to pursue my \gls{Ph.D.} research goals with great dedication. I think this journey report can be helpful for the computing community in promoting education more fairly and responsibly. I did my best, and for this reason, I am sure that I explored my available capabilities and put them in the service of the research community. I finish this formation cycle with joy and a bit of tiredness, but also with a sense of accomplishment.

  I have a dream that, one day, equity analysis will be an integrated part of \gls{CEd} formative assessment. Computing exists because people exist. People are complex and need to be considered in a systemic approach where possible. May our common goal be for a more humanized \gls{CEd}.


