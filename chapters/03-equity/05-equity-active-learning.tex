\section{Equity and Active Methodologies in CEd}
\label{equity-sec:active-learning}

Parents tackle some situations to guarantee an environment with fewer disputes among their children. Imagine the scenario where we have a parent, two children, and a bar of chocolate. This parent has the following question: How to teach your children to find the best way to divide a bar of chocolate between them?

A possible way to solve this problem is this parent follows the approach: (i) dividing the bar into two equal parts, and (ii) giving each part to each one.  Although it seems fair, this parent would make the whole process and not allow the children to learn in this situation. In addition, (s)he would be the ruler and be obliged to handle possible objections of a wrong division made in step (i).
This parent can use an alternative way to solve this problem. There is an approach called “cut-and-solve”. It consists of just two steps: (i) one child cuts the bar, and (ii) another one chooses a piece. Thus there are a divider and a chooser. The divider will not choose, and the chooser will not divide. This approach is a good idea because this parent solves one drawback: (s)he would not be the divider (because that who divides can fail to perform a precise division). But an important something remains: (s)he has been the ruler yet.

The first way seems more authoritarian, although it attends to fair principles. But the alternative way seems more respectful and guarantees other things beyond fairness. This guarantees (i) the reflection of equitable variables trade-off, (ii) a discovery learning solution, and (iii) the promotion of democratic values. The “cut-and-solve” approach is an old (and an excellent) solution for some cases of simple fair division problems \cite{brams:2020}. Academic papers have formally addressed this issue since 1948 (e.g., \citeonline{steinhaus:1948}).

In \gls{CEd}, several works approach equity in active learning contexts from pair programming use. \citeonline{lewis:2015}, for instance, identified patterns of marginalization and domination between the pairs. Beyond dominance \cite{grabl:2024}, more recent papers investigated other equity issues like gender \cite{bodaker:2023}, refugees \cite{arawjo:2021}, and sense of belonging \cite{izhikevich:2022}.

I explored other challenges in this subject through a fictitious story\footnote{This story was written originally in Brazilian Portuguese in this essay \cite[pp.~277,278]{bispojr:2022-educomp}.}: 
\begin{citacao}
    "Professor Quincas Borba is very happy with the use of active approaches. He realizes that there is more enthusiasm in his classroom in a general way. Several students like to attend and participate in his classes. He also realizes that many of his students can solve real problems involving data structures in a way that the proportion of competent students to do this is higher than before when he used the expositive format more strongly in his classes. The students complained about the higher workload that they need to do now, but, in a general way, they approved the changes made by him.

    However, Prof. Quincas is disturbed by a specific scenario in his classroom. There is a group of students that is not adapting well to this new approach. Among these students, two are of special interest: Lindoia and Brás. Lindoia is a joyful student and looks to demonstrate a certain interest in her studies. However, she is a little introverted. She entered the Computing program by affirmation policies as an indigenous person. Brás is a guy who clearly demonstrates his desire to learn just by seeing his facial expression. He has an easy smile and likes to be quite cordial. He does classroom activities always together with his faithful partner, Capitu, his sign-language interpreter. He is deaf-mute.

    Prof. Quincas finds himself in a quite hard and challenging situation. His active methodologies require a certain dynamism in the activity conduction. This rhythm is necessary for the moments of group discussion, alternating with individual and gamification moments, which play the role of involving the students in a favorable and stimulating learning environment. However, Lindoia and Brás have many difficulties in actively participating in activities. Due to both situations, the language flow is not performed satisfactorily because Lindoia and Brás do not have Portuguese as their native language but as a second one. So, in activities in which Lindoia and Brás participate, they always leave behind flux, as suggested by Prof. Quincas. They make an effort to follow activities. Lindoia makes a higher cognitive effort to live together in an educational space where everybody does not speak Guarani\footnote{Indigenous language from the South part of South America spoken by people from Tupi-Guarani ethnicity.}. Brás keeps his fingers crossed for the effort and dedication of Capitu can, as quickly as possible, understand the speech content of his colleagues and Prof. Quincas and make the interpretation in Brazilian Sign Language for him (and vice versa). As the dynamic cools down in scenarios like this, Prof. Quincas also realizes that there is no natural disposition of other students to want to participate in groups when Lindoia and Brás are present".
\end{citacao}
These challenges need to be more investigated in \gls{CSE} contexts, mainly in Brazilian ones. \citeonline{murray:2024} reported an excellent qualitative study from \gls{SLICC}, but in an interdisciplinary way and not focused on \gls{CEd}.

% \vspace{0.3cm}

% \fbox{
%     \begin{minipage}[htb]{0.9\textwidth}
%         \vspace{0.3cm}
                
%         \colorbox{gray!30}{% create a colored box
%             \makebox[0.975\textwidth][l]{% center the text on the page
%                 \ \ \textbf{Further Writing}
%             }
%         }

%         \vspace{0.1cm}
        
%         \begin{itemize}
%             \item Presenting the \citeonline{lewis:2015} paper.
%             \item Including that fictitious story created by me \cite{bispojr:2022-educomp}.
%             \item Presenting the potential of active methodologies to promote social justice.
%             \item Highlighting that SDL is weakly investigated in this direction.
%         \end{itemize}

%         \vspace{0.25cm}
        
%     \end{minipage}
% }
