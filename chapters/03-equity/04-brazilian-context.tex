\section{Brazilian Context of Higher Education}
\label{equity-sec:br-context}

In 2022, Brazil offered more than 22 million vacancies in higher education. Only 3.81\% of these vacancies were in the public educational system, revealing that 74.74\% of more than 22 million vacancies referred to online education in the private educational system. When we restrict these vacancies to in-person higher education (more than 5 million), this percentage increases to 13.48\% in the public educational system, still indicating the scarcity of the public good that is the access to public higher education at Brazil\footnote{Data collected from the presentation for the press conference of \gls{INEP} about the 2022 Brazilian Higher Education Census (Slide 20). See more in the Brazilian Portuguese reports available in \url{https://www.gov.br/inep/pt-br/areas-de-atuacao/pesquisas-estatisticas-e-indicadores/censo-da-educacao-superior/resultados}.}.

High school graduates are natural candidates to dispute these higher education vacancies. In 2023, 83.56\% of Brazilian fresh students in high school were in public educational institutions, whereas 16.44\% were in private ones\footnote{Data collected from the presentation for the press conference of \gls{INEP} about the 2023 Brazilian School Census (Slide 26). See more in the Brazilian Portuguese reports available in \url{https://www.gov.br/inep/pt-br/areas-de-atuacao/pesquisas-estatisticas-e-indicadores/censo-escolar/resultados}.}. It is probable that the most part of higher education candidates are from public educational system but, intriguingly, the offering of higher education vacancies in this public system is scarce. Thus, it seems that Brazilian public higher education is a not-abundant common good, being necessary to create ways of how the society could better benefit from it.

Another critical perspective concerning income distribution in Brazil \cite{sasse:2021}. Only 1\% of richest people holds on 23.3\% of national income, while 40\% of poorest one holds only 10.4\%. This means that the 1\% of  the richest parcel of Brazilian population has the double of income of 40\% of poorest one. These values puts Brazil at the second place on the list of 180 countries with more income concentration in the world (only behind Qatar).

Bearing this scenario in mind, since 2013, the Brazilian Ministry of Education\footnote{See more detail in the \gls{MEC} official site: \url{http://portal.mec.gov.br/cotas/sobre-sistema.html}.} adopts system of quotas in all federal educational institutions. One of the principles is avoiding what \citeonline[p.~32]{bourdieu:2013} called "school as a conservative force", giving crucial steps for a more equitable educational system that can promote the school as a factor of social mobility. In Brazilian federal higher institutions, 50\% of vacancies are reserved for high school graduate candidates from public institutions. Inside each half of vacancies, other criteria apply in this order: \gls{HPCI}, race/ethnicity, and disableness. The 2023 Brazilian Higher Superior Census, concerning federal students, reveals that 51\% that came from the system of quotas finished their undergraduate studies, while this rate is 41\% from other students\footnote{Available as news from \gls{INEP} site: \url{https://www.gov.br/inep/pt-br/assuntos/noticias/censo-da-educacao-superior/mec-busca-garantir-permanencia-de-estudantes-mais-vulneraveis}.}.